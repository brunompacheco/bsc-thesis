% ----------------------------------------------------------
\chapter{Deep Learning}\label{cap:deep-learning}
% ----------------------------------------------------------

In this chapter, a brief overview of the foundations of deep learning will be presented. As deep learning is such a broad area of knowledge, this chapter will focus on the subjects related to the thesis' objectives. At the same time, even though physics-informed learning and implicit models can be seen as branches of deep learning, they will be introduced in separated chapters for their relevance to the presented work.

% ----------------------------------------------------------
\section{Origin/Introduction/Definition?}
% ----------------------------------------------------------

Even though \textit{Deep Learning} may seem like a novel and exciting technology, it has been studied under many different names since the 1940s \cite{goodfellow_deep_2016}.
Deep learning involves composing multiple levels\footnote{The understanding of these levels as a learning \textit{depth} is the origin of the naming for deep learning.} of representation learning.
Representation learning, in turn, is about automatically extracting higher level features from the input data\cite{lecun_deep_2015,bengio_representation_2013}, aiming to facilitate the extraction of useful information, for example, to classify the data into given categories.

Following these definitions, we can see a deep learning \textit{model} as a function with multiple levels of representations.
A simple way of putting this definition into terms would be to define a model with $L$ layers as a function $f$ such that, for a given input $x$,
\begin{align*}
    z^{[0]} &= x \\
    z^{[i]} &= f^{[i]}(z^{[i-1]}),\,i=1,\ldots,L \\
    f(x) &= z^{[L]}
.\end{align*}
From this notation, it is easy to see that each function $f^{[i]}$ maps the outputs of the previous layer into the input of the next, ideally achieving a higher abstraction level.

In this context, we can see the goal of \textit{learning} as finding parameters $\theta$ such that the layers indeed extract the desired information, culminating in the model returning the desired output.
A classic toy example in deep learning models is to approximate the Exclusive-OR function, that is, to find a model $f : \R^2 \to \R$ such that
\begin{align*}
    f([0,0]^T) = f([1,1]^T) = 0 \\
    f([1,0]^T) = f([0,1]^T) = 1 \\
.\end{align*}
We can construct this model's layers as functions $f^{[1]}:\R^2\to\R^2$ and $f^{[2]}:\R^2\to\R$ such that
\begin{align*}
    f^{[1]}(\bm{z}) &= \begin{bmatrix}
    OR(\bm{z}) \\
    NAND(\bm{z})
    \end{bmatrix} \\
    f^{[2]}(\bm{z}) &= AND(\bm{z})
.\end{align*}
Note how these functions are extracting simple information from the input, yet their stacking can approximate very well the desired behavior.

\section{Deep Feedforward Networks}

Deep Feedforward Networks are the most essential deep learning models.
They were inspired by the works of \textcite{rosenblatt_perceptron_1957} and refined through the many decades since, culminating in one of the most important deep learning models for practitioners and the basis of of the most proeminent results seen in the recent years\cite{goodfellow_deep_2016}.

One can say that the goal of a feedforward network is to approximate a given function through a mapping from a set of parameters and input data.
The way this is achieved is through the stacking of simple linear combinations followed by a nonlinear \textit{activation} function, i.e., following the notation previously presented, we can define a deep feedforward neural network with $L$ layers as a \textit{parametrized} function $f_\theta$ such that, for a given input $\bm{x}$,
\begin{align*}
    \bm{z}^{[0]} &= \bm{x} \\
    \bm{z}^{[i]} &= f_\theta^{[i]}(\bm{z}^{[i-1]}) = g^{[i]}\left(A^{[i]}\bm{z}^{[i-1]} + \bm{b}^{[i]}\right) ,\,i=1,\ldots,L \\
    f_\theta(\bm{x}) &= \bm{z}^{[L]}
,\end{align*}
in which $\theta$ is the set of parameters and $g^{[i]}$ are the activation functions. In here, the parameters can be described as $\theta = \left(A^{[1]},\bm{b}^{[1]},\ldots,A^{[L]},\bm{b}^{[L]}\right)$. Using this notation, we can say that, given a \textit{target} function $f^*$ and a set of input data $X$, a deep feedforward network $f_\theta$ must learn a set of parameters $\theta$ such that $f_\theta(x) \approx f^*(x),\,\forall x \in X$.

Let us recall the Exclusive-OR example from the previous section. One can construct a two-layer deep feedforward network such that the functions $f_\theta^{[1]}:\R^2\to\R^2$ and $f_\theta^{[2]}:\R^2\to\R$ are
\begin{align*}
    f_\theta^{[1]}(\bm{z}) &= \sigma\left(
    \begin{bmatrix}
    2K & 2K \\
    -2K & -2K
    \end{bmatrix}\bm{z} + \begin{bmatrix}
    -K \\
    3K
    \end{bmatrix}\right) \\
    f_\theta^{[2]}(\bm{z}) &= \sigma\left(
    \begin{bmatrix}
    2K & 2K
    \end{bmatrix}\bm{z} - 3K\right)
,\end{align*}
where the chosen activation function $\sigma(.)$ is the sigmoid function\footnote{Applied element-wise where necessary.} and the parameters are defined from $K \gg 1$ such that $\sigma(K) \approx 1$ and $\sigma(-K) \approx 0$.
Then, it is easy to see that the first output of $f^{[1]}$ approximates the OR function applied to the input, while the second output approximates the NAND function and $f^{[2]}$ approximates an AND function.

For a network defined as previously, we say that the inner states $\bm{z}^{[i]}$ which are neither the input nor the output of the network (i.e., $i\not\in \{0,L\}$) are called the \textit{hidden layers} of the network.
Note that even though the input and output dimensions are defined by the target function, the dimensions of the hidden layers are a design choice, as well as the number of hidden layers and the activation functions.

It has already been proven that a network with a single hidden layer can approximate any continuous function on a closed and bounded subset of $\R^n$, for a broad range of activation functions, given that the hidden layer has enough dimensions\cite{hornik_multilayer_1989,leshno_multilayer_1993}.
Yet, even though this universal approximation theorem guarantees that such a network exists, it provides no way to find it.
In practice, a network with a single hidden layer might need to be unfeasibly large to achieve the desired approximation, while deeper models can be much more efficient\cite{goodfellow_deep_2016}.

\section{Learning}

The Exclusive-OR example illustrates well how a model built of multiple simple functions can approximate very well a target behavior.
Yet, if we consider complex tasks (such as predicting age from humans' photographs), it is easy to see that many components and hundreds, maybe millions of parameters may be necessary\footnotemark.
Thus, it is not always reasonable (or even feasible) to design these components manually.
That is the reason \textit{learning algorithms} are essential to make these models useful for realistic tasks.

\footnotetext{A simple example would be to consider a deep feedforward network designed to have images as inputs and output a single value. Even if small, 32-by-32 pixels, grayscale images are expected, the domain will have dimension 1024. Thus, if a single hidden layer of dimension 256 (a quarter of the input size) is desired before the output layer, the network will have over 200 thousand parameters.}

One can understand what a learning algorithm is from the definition from \textcite{mitchell_machine_1997}:
"A computer program is said to learn from experience $E$ with respect to some class of tasks $T$ and performance measure $P$, if its performance at tasks in $T$, as measured by $P$, improves with experience $E$".
To narrow down this broad definition to the scope of the presented work, we can consider the tasks in $T$ as target functions which we want our model to approximate.
For this class of tasks, it is natural that the performance measure $P$ be some sort of distance measure between the model and the target function.

It is often impractical to evaluate both the model and the target function in the entirety of the domain desired.
Both because it can be present high computational costs, and because the behavior of the target functions might not be known beforehand.
Thus, the experience $E$ usually comes from data in the form of a (finite) set of samples from the domain paired with the outputs of the target function of interest.
% The performance is also usually measured on the data, discarding the need to evaluate the target function and the model on the entirety of the domain.

Roughly speaking, the deep learning algorithms of interest for this work are those that, given data on the target function, provide a deep learning model that maximizes a performance measure.

\subsection{Gradient Descent}

The majority of deep learning algorithms of our interest require some sort of optimization.
It would be natural to frame the definition of a learning algorithm as an optimization problem that maximizes the performance measure.
Yet, not always the performance measure of interest is easy to compute or provides useful features for the optimization (e.g, differentiability).
Therefore, it is usual to optimize indirectly, minimizing a \textit{loss function} while aiming to improve the performance measure.

Given a model $f_\theta:\R^n\to\R$ and a target function $f^*:U\subset\R^n\to\R$, an ideal performance measure could be $\int_U |f_\theta(\bm{x}) - f^*(\bm{x})|dx$.
Yet, this integral could be costly to compute and might not even be easily defined.
Instead, one could evaluate the model on a finite set of points from the domain $X=\{(\bm{x},y)\in U\times\R : y=f^*(\bm{x})\}$ (the data that forms the experience $E$), defining a loss function $l:\R\times\R\to\R$ that can be evaluated per sample, e.g., the $\ell^2$-norm.
This way, the return of the deep learning algorithm would be a model $f_{\theta^*}$, such that \[
\theta^* = \arg\min_\theta \sum_{(\bm{x},y)\in X} l(f_\theta(\bm{x}), y)
,\] therefore, one can solve the optimization problem
\begin{align*}
    \min_\theta \quad & \sum_{(\bm{x},y)\in X} l(f_\theta(\bm{x}), y) \\
    \text{s.t.} \quad & \theta \in \Theta
,\end{align*}
where $\Theta$ is the set of feasible parameters for the model.

The most common way to solve this optimization for deep learning models is to use the \textit{gradient descent} algorithm.
This method was first proposed by Cauchy in the XIX century\cite{lemarechal_cauchy_2012} based on the definition of the gradient of a differentiable function.
It is known that, given a differentiable function $f:A\to\R$ and $\bm{a} \in A$, if $\| \nabla_{\bm{a}} f(\bm{a}) \| \neq 0$\footnotemark, $\exists \gamma > 0$ such that $f(\bm{a} - \gamma \nabla_{\bm{a}} f(\bm{a})) < f(\bm{a})$, that is, if one takes a small enough step in the opposite direction of the gradient, the function is certain to decrease.
The scalar $\gamma$ is called the \textit{learning rate}, and defines the size of the step that is taken\cite{goodfellow_deep_2016}.
\footnotetext{The notation $\nabla_{\bm{a}} f$ represents the gradient of $f$ \textit{with respect to} the input $\bm{a}$, which could be omitted as the functions has only this input.}

Therefore, following the notation of the previous example, if we assume that both the model and the loss function are differentiable, for a given pair $(\bm{x},y)$ and parameters $\theta_k$ such that $\| \nabla_\theta l(f_{\theta_k}(\bm{x}),y) \| \neq 0$, if we set \[
\theta_{k+1} \gets \theta_k - \gamma \nabla_\theta l(f_{\theta_k}(\bm{x}),y) \tag{$*$}\label{eq:gradient-descent-step}
,\] then, for a sufficiently small $\gamma$, we know that \[
l(f_{\theta_{k+1}}(\bm{x}),y) < l(f_{\theta_k}(\bm{x}),y)
,\] i.e., the loss will decrease.
Notice how it seems that if one takes enough steps like \eqref{eq:gradient-descent-step} with small enough $\gamma$, the parameters will converge to a point that is a local minimum of the loss function.

% Maybe introduce the gradient step with regards to the cost function = sum of per-sample loss function?

Yet, gradient descent provides poor convergence conditions\cite{wolfe_convergence_1969}, being considered unreliable and slow for many practical problems.
Still, gradient descent is known to work very well for deep learning.
In this area, it has shown to low values of the loss function fast enough to be useful, even if it does not find a local minimum.
\cite{goodfellow_deep_2016}

\subsection{Gradient Descent Variations}

Over the year, many improvements on the gradient descent algorithms made it even more reliable and efficient for learning deep networks.
In particular, the use of \textit{momentum} on the update of the parameters.
Computing a velocity vector over the parameter update steps, and taking this into account when performing the update, has shown to accelerate significantly the learning[REFTO Sutskever, 2013].
The classical way of doing this is to replace the update in \eqref{eq:gradient-descent-step} by
\begin{align*}
    v_{k+1} &\gets \mu v_k - \gamma \nabla_\theta l(f_{\theta_k}(\bm{x}),y) \\
    \theta_{k+1} &\gets \theta_k + v_{k+1}
,\end{align*}
where $v_k$ is the velocity at the $k$ step, and $\mu$ is the momentum coefficient.

A major challenge of the gradient descent as a learning algorithm is that it introduces a parameter that is not learned in the process (which are commonly called \emph{hyperparameters}) and that is crucial to finding a good model: the learning rate.
Using momentum reduces a little the sensitivity of the results to the choice of the learning rate, but does so while introducing a new hyperparameter (the momentum coefficient $\mu$).
Furthermore, it is known that the loss function can be sensitive to some of the parameters while being numb to others \cite{goodfellow_deep_2016}.

It was natural, in the face of these challenges, to think of strategies that use different learning rates for each parameter and automatically change these learning rates throughout the learning process.
The most simple way of doing this is following the idea that, if the gradient with respect to a given parameter does not change sign, its associated learning rate should increase; otherwise, it should decrease.
This way, if the loss function is sensitive to a given parameter, its associated learning rate will decrease over time, making the training less unstable.

Adam[REFTO Kingma, 2015] combined two impactful implementations [REFTO Duchi, 2011; Tieleman_Hinton, 2012], encompassing both adaptive learning rates and momentum and taking them one step further.
For this reason, Adam is one of the most commonly used algorithms and is seen as one of the best choices overall [REFTO Ruder, 2016].

\section{Regularization}

