% ----------------------------------------------------------
\chapter{Demais seções (capítulos)}
% ----------------------------------------------------------

\textbf{Instruções da Coordenação do PFC:}

Uma prática que contribui para uma fuidez melhor do texto é colocar um parágrafo introdutório no início de cada capítulo, descrevendo os assuntos que serão abordados e a relação com o restante do trabalho. Por exemplo, ``A Seção 2.1 apresenta \ldots'', ``Os resultados obtidos são analisados na Seção 2.2.'' Pode-se fazer o mesmo no início de seções maiores, explicando para o leitor, em uma ou duas sentenças o que está por vir no texto e por quê. Outra boa prática é, ao final de cada capítulo, fazer a ligação com o capítulo seguinte.

Figuras, tabelas, quadros e equações devem ser introduzidos e explicados no texto; não se pode simplesmente ``jogá-los'' no texto, sem referência nem explicação). Por exemplo, escreva: ``O circuito projetado é mostrado na Figura 1. O resistor $R_1$ faz o papel de um limitador de corrente, enquanto o capacitor $C_2$ juntamente com o resistor $R_5$ formam um filtro passa-baixa. Este circuito tem a vantagem de \ldots''

Com relação às equações, não se faz referência a uma equação que ainda não foi apresentada. Por exemplo, não se escreve: ``A relação entre a tensão e a corrente de um resistor é dada pela \autoref{eq:leiDeOhm}'':

\begin{equation}
    V = R I \, \text{.}
    \label{eq:leiDeOhm}
\end{equation}

\noindent O correto é algo como ``A relação entre a tensão e a corrente de um resistor é dada pela Lei de Ohm,

\begin{equation}
    V = R I \, \text{,}
    \label{eq:leiDeOhm2}
\end{equation}

\noindent na qual $V$ é a tensão aplicada no resistor, $R$ é a resistência e $I$ é a corrente elétrica''.

É importante observar que as equações fazem parte do texto e, assim, frequentemente convém inserir uma vírgula ou ponto ao seu final. Se o parágrafo segue, elimina-se o recuo na próxima linha com o comando \verb!\noindent!. Além disto, se a frase segue, inicia-se a linha com letra minúscula. Veja exemplos na \autoref{eq:leiDeOhm} e \autoref{eq:leiDeOhm2}.



A seguir encontra-se uma equação na linha de texto: $\hat{y}(t+k\mid t)= \sum^\infty_{i=1} g_i \Delta u(t+k-i\mid t)$. E eis uma referência cruzada da \autoref{fig:Fig_1} e da \autoref{eq:Eq_1}.