% ----------------------------------------------------------
\chapter{Experiments and Results}\label{ch:experiments}
% ----------------------------------------------------------

Given the theoretical foundations exposed above, in this chapter, the proposed approach to achieve this thesis' goal is presented.
Furthermore, this approach named \gls{PIDEQ} is validated through a set of experiments, aiming to explore its performance and the impact of the many different features of the model.

\section{Problem Definition}

The Van der Pol oscillator (sec. \ref{sec:vdp}) was chosen as the \gls{ODE} system for which an \gls{IVP} will be solved.
This system is known for having no analytical solution, thus becoming a benchmark for solvers.
As well, textcite[REFTO Antonello, 2021] have already studied a solution using \gls{PINN}s, which is used as a starting point and a reference in performance for the experiments.

More specifically, we define the first-order formulation of the Van der Pol oscillator (as presented in equation \eqref{eq:vdp}) as the \gls{ODE} system, with $\mu=1$.
Then, the \gls{IVP} is defined with initial condition $\bm{y}(0) = \bm{y}_0 = \left( 0, 0.1 \right) $, simulating a small perturbation of the system at the unstable equilibrium at the origin, and the solution is desired for a horizon of 2 seconds. 
This way, the expected solution is expected to gravitate to a limit cycle, but not within the solution interval, as illustrated in figure \ref{fig:vdp_example}.

\subsection{Evaluation Metrics}

As there is no analytical solution to the \gls{IVP} at hand, the solutions will be evaluated in comparison to the approximation found using \gls{RK4}.
This reference was generated from 1000 points equally spaced in the solution interval $I=[0,2]$, with a time step of 2\,ms\footnotemark.
\footnotetext{Of course, this set does not include the initial condition, which is already given, which means that the first point of evaluation is at $t=0.002$.}
Then, a solution's approximation to the reference is measured through the \gls{IAE}, which can be defined here as \[
    IAE = \frac{1}{h}\sum_{i=1}^{1000} \|\bm{y}_i - \hat{\bm{y}}_i\|
,\] where $\bm{y}_i$ are the points in the \gls{RK4} solution,  $\hat{\bm{y}}_i$ are the points in the solution being evaluated, and $h=0.002$ is the time step.
Therefore, a solution is said proper to the problem if it achieves a low \gls{IAE}.

Besides the quality of the approximation, the time required to achieve such approximation is also of interest, as many applications (e.g., model predictive control) are highly time-dependent.
As well, the hardware resources used are valuable metrics, as the computing power necessary limits the range of equipment that can support a given solution.
These will be auxiliary to the \gls{IAE} in the analysis of the approximations.

\section{PIDEQ}

As already discussed in section \ref{sec:pinn-problem}, solving \gls{IVP}s is (mostly) only reasonable if using a physics-informed approach, that is, if "teaching" the model through the known dynamics instead of through actual samples of the target function.
Therefore, a reasonable solution using \gls{DEQ}s must follow the same principle, which implies in a physics-informed training of \gls{DEQ}s (thus, \gls{PIDEQ}).

For the \gls{IVP} defined above, we recall the definition of section \ref{sec:deq-definition} and propose a \gls{DEQ} similar to textcite[REFTO Ghaoui, 2019], that is, a model
\begin{align*}
    D_{\gls{param}}^{EQ}: \R &\longrightarrow \R^2 \\
    t &\longmapsto D_{\gls{param}}(t) = \hat{\bm{y}}
\end{align*}
such that
\begin{equation}
\begin{split}
    D^{EQ}_{\gls{param}}(t) &= C\bm{z}^{\star} \\
    \bm{z}^{\star} &= \bm{f}_{\gls{param}}\left( t,\bm{z}^{\star} \right) \\
    \bm{f}_{\gls{param}}\left( t,\bm{z} \right) &= \tanh \left( A\bm{z} + t\bm{a} + \bm{b} \right)
\end{split}
,\end{equation}
in which the parameters \gls{param} are a vectorization of the matrices and vectors, i.e., $\gls{param} = \left( A,C,\bm{a},\bm{b} \right)$, and the hyperbolic tangent function is applied to each element of the resulting vector.

Notice that this formulation is very close to the one used in chapter \ref{ch:deq}, except for the linear transformation of the equilibrium point at the model's output.
This implies that both forward and backward operations can occur in the same way, except that the term  \[
    \frac{d D^{EQ}_{\gls{param}}}{d \bm{z}^{\star}} = C^T
\] must be multiplied in the computation of the gradients.
The advantage of this modification is that we can $\bm{z}$ with an arbitrary dimension, which results in arbitrary representational power [REFTO Ghaoui, 2019].

- cost function definition


\section{Experiments}

\subsection{Data}

- Following the setup of [REFTO Antonelo] 
- description of collocation points
- data points are just the initial condition

\subsection{Training Setup}

- the experiments reported below have a set of hyperparameters that were defined from early experimental results
- report hyperparameters
- all results were generated in a (INSERT HARDWARE)

\subsection{Baseline Results}

- Antonelo trained a PINN to solve this exact same problem with 4x20
- we replicate this results
- theoretical baseline is that a DEQ has at least as much representational power as a net with the same amount of nodes in total, so a DEQ with 80 states should be able to learn the task
- show results

\subsection{Number of States}

- show sparse matrix and how it points out to a smaller model
- process repeated
- while a 5 states DEQ can learn the task, it takes longer to converge


