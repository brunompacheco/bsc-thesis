% ----------------------------------------------------------
\chapter{Ordinary Differential Equations}
% ----------------------------------------------------------

This first chapter is a short review of \gls{ODE}s and how to solve them numerically.
It would be unreasonable to assume that one chapter of a bachelor thesis could cover what many authors have dedicated entire books to communicate.
Therefore, it is assumed that the reader has a basic (undergraduate level) understanding of the theory of differential equations\footnotemark.
\footnotetext{Otherwise, the books of \textcite{zill_first_2013} and \textcite{simmons_differential_2017} are great resources.}
This way, this chapter focuses on reviewing the fundamentals and establishing a notation concise for the following chapters, besides introducing the approach to numerical solve \gls{ODE}s.

\section{Introduction and Definitions}

A differential equation can be seen as a description of the relationship between unknown quantities and their rates of changes.
Because of this broader definition and direct relationship to applications, differential equations arise naturally in many fields of applied sciences\cite{hairer_solving_1993}, as it is often the case that the rate of change of a certain quantity of interest is related to the rate of change of other quantities.
A classical example of a differential equation is Newton's second law of motion \[
F\left( x\left( t \right) \right) = m \frac{d^2 y(t)}{d t^2}
,\] in which $x\left( t \right) $ is the position of an object at time $t$, $m$ is its mass, and $F(x)$ is the force under which the object is at a given position.
This example highlights one of the great powers of the differential equations, which is to describe the dynamics of a certain phenomenon without explicitly defining it.


For a more tangible definition, any equation that contains the derivatives of (at least) one unknown function with respect to (at least) one independent variable is called a \emph{differential equation}\cite{zill_first_2013}.
A differential equation that involves only \emph{ordinary} derivatives, that is, only derivatives of functions with respect to a single variable, (such as the one above) is called an \gls{ODE}.
% A differential equation can also be classified through its \emph{order} (the order of its highest derivative)
\gls{ODE}s can be represented in the normal form
\begin{equation}\label{eq:ode}
    \frac{d^n \bm{y}(t)}{d t^{n}} = \mathcal{N}\left( t, \bm{y}\left( t \right), \frac{d \bm{y}(t)}{d t}, \ldots,\frac{d^{n-1}\bm{y}(t)}{d t^{n-1}} \right)
\end{equation}
%\gls{ODE}s can be represented in the normal form \[
    %\frac{d^n \bm{y}(t)}{d t^{n}} = \mathcal{N}\left( t, \bm{y}\left( t \right), \frac{d \bm{y}(t)}{d t}, \ldots,\frac{d^{n-1}\bm{y}(t)}{d t^{n-1}} \right) \tag{$*$}\label{eq:ode}
in which $\bm{y}\left( t \right) $ is a vector-valued continuous function, $\mathcal{N}:\R\times \R^{n}\to \R$ is a real-valued continuous function, and $n$ is the order of the highest derivative in the equation and is commonly referred to as the \emph{order of the differential equation}.

\section{Initial Value Problems}

It is a common problem to find an explicit definition of the unknown functions in a differential equation.
Given an $n$-th order ODE such as \eqref{eq:ode}, any function $\bm{\phi}:I\subset\R\to \R^{m}$ is said to be a \emph{solution} of this equation if it satisfies \[
    \frac{d^n \bm{\phi}(t)}{d t^{n}} = \mathcal{N}\left( t, \bm{\phi}\left( t \right) , \frac{d \bm{\phi}(t)}{d t}, \ldots,\frac{d^{n-1}\bm{\phi}(t)}{d t^{n-1}} \right),\,\forall t\in I
.\] Note, however, that the solutions are not necessarily unique.
As an example, given Newton's second law of motion with constant force ($F(x)=C$), then it is easy to see that any second-order polynomial of the form \[
    \phi\left( t \right) = \frac{C}{2m}t^2 + a_1t + a_0,\, a_0,a_1\in \R
\] is a solution.

Yet, for the case of ODEs, it is often the case that if one has $n$ side conditions on the (lowest order) derivatives of the unknown function the solution exists and is unique, i.e., for an ODE defined as in \eqref{eq:ode}, conditions of the form \[
\bm{y}\left( t_0 \right) =\bm{y}_0,\frac{d\bm{y}\left( t_1 \right) }{dt_1}=\bm{y}_1,\ldots,\frac{d^{n-1}\bm{y}\left( t_{n-1} \right) }{dt^{n-1}}= \bm{y}_{n-1}
,\] in which $t_0,\ldots,t_{n-1}\in I$ and $y_0,\ldots,y_{n-1}\in \R^m$, guarantee the uniqueness of the solution in an interval $I$ if $\mathcal{N}$ is Lipschitz continuous\footnote{Lipschitz continuity is a stronger form of continuity, with a certain limit to a function's rate of change. Refer to \textcite{sohrab_basic_2003} for a proper definition.} on a subset of $\R\times \R^{n}$\footnotemark. \cite{coddington_theory_2012}
\footnotetext{This is a brief and rude description of the Picard-Lindelöf theorem, which actually only guarantees that the solution exists and is unique on neighborhoods of $t_i$. Yet, in many practical applications it is the case that a unique solution exists for the trivial subsets of $\R$.}
The problem of finding the solution given conditions as above is called the \gls{IVP}.

\gls{IVP} shows up frequently when the current (or past) state of a system is known and the future state is desired. Recalling Newton's second law example, suppose that, besides $F(x)=C$, it is also known that the object lies still at $t=0$, i.e., \[
y(0) = 0,\,\frac{d y(0)}{dt}=0
,\] and we are interested in the solution in the interval $I=\left[ 0,T \right] $, for $T>0$. Then, the only solution is that with $a_0=a_1=0$, that is, \[
    \phi\left( t \right) = \frac{C}{2m}t^2
.\] 

\section{Applications}

- why are ODEs useful?
- [intersection with control and automation] 

\subsection{Van der Pol Oscillator}

\section{Solvers}

