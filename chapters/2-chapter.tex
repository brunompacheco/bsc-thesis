% ----------------------------------------------------------
\chapter{Ordinary Differential Equations}
% ----------------------------------------------------------

This first chapter is a short review of \gls{ODE}s and how to solve them numerically.
It would be unreasonable to assume that one chapter of a bachelor thesis could cover what many authors have dedicated entire books to communicate.
Therefore, it is assumed that the reader has a basic (undergraduate level) understanding of the theory of differential equations\footnotemark.
\footnotetext{Otherwise, the books of \textcite{zill_first_2013} and \textcite{simmons_differential_2017} are great resources.}
This way, this chapter focuses on reviewing the fundamentals and establishing a notation concise for the following chapters, besides introducing the approach to numerical solve \gls{ODE}s.

\section{Introduction and Definitions}

A differential equation can be seen as a description of the relationship between unknown quantities and their rates of changes.
Because of this broader definition and direct relationship to applications, differential equations arise naturally in many fields of applied sciences\cite{hairer_solving_1993}, as it is often the case that the rate of change of a certain quantity of interest is related to the rate of change of other quantities.
A classical example of a differential equation is Newton's second law of motion \[
F\left( x\left( t \right) \right) = m \frac{d^2 y(t)}{d t^2}
,\] in which $x\left( t \right) $ is the position of an object at time $t$, $m$ is its mass, and $F(x)$ is the force under which the object is at a given position.
This example highlights one of the great powers of the differential equations, which is to describe the dynamics of a certain phenomenon without explicitly defining it.


For a more tangible definition, any equation that contains the derivatives of (at least) one unknown function with respect to (at least) one independent variable is called a \emph{differential equation}\cite{zill_first_2013}.
A differential equation that involves only \emph{ordinary} derivatives, that is, only derivatives of functions with respect to a single variable, (such as the one above) is called an \gls{ODE}.
% A differential equation can also be classified through its \emph{order} (the order of its highest derivative)
\gls{ODE}s can be represented in the normal form \[
    \frac{d^n \bm{x}(t)}{d t^{n}} = \mathcal{N}'\left( t, \bm{x}\left( t \right), \frac{d \bm{x}(t)}{d t}, \ldots,\frac{d^{n-1}\bm{x}(t)}{d t^{n-1}} \right)
.\] 
in which $\bm{x}:\R\to \R^{m'} $, $\mathcal{N}':\R\times \R^{nm'}\to \R^{m'}$ is a continuous function, $\frac{d^n \bm{x}(t)}{d t^{n}}$ is the vector containing the derivatives with respect to $t$ of all $\bm{x}\left( t \right) $ components\footnotetext{I.e., the first (and only) row-vector of the Jacobian of $\bm{x}\left( t \right) $.}, and $n$ is the order of the highest derivative in the equation and is commonly referred to as the \emph{order of the differential equation}.

Any function $\bm{\phi}:I\subset\R\to \R^{m'}$ is said to be a \emph{solution} of an $n$-th order \gls{ODE} if it satisfies \[
    \frac{d^n \bm{\phi}(t)}{d t^{n}} = \mathcal{N}\left( t, \bm{\phi}\left( t \right) , \frac{d \bm{\phi}(t)}{d t}, \ldots,\frac{d^{n-1}\bm{\phi}(t)}{d t^{n-1}} \right),\,\forall t\in I
.\] Note, however, that the solutions are not necessarily unique.
As an example, given Newton's second law of motion with constant force ($F(x)=C$), then it is easy to see that any second-order polynomial of the form \[
    \phi\left( t \right) = \frac{C}{2m}t^2 + a_1t + a_0,\, a_0,a_1\in \R
\] is a solution.

Now, given an $n$-th order \gls{ODE}, let $\bm{y}_1\left( t \right) =\bm{x}\left( t \right),\,\bm{y}_2\left( t \right) = \frac{d \bm{x}\left( t \right) }{d t} ,\ldots, \bm{y}_n\left( t \right) = \frac{d^{n-1} \bm{x}\left( t \right) }{d t^{n-1}}$.
Note that we can now write the following \emph{system} of first-order differential equations
\begin{align*}
    \frac{d \bm{y_1}\left( t \right) }{dt} &= \frac{d \bm{x}\left( t \right) }{d t} = \bm{y}_2\left( t \right) \\
    &\vdots \\
    \frac{d \bm{y_{n-1}}\left( t \right) }{dt} &= \frac{d^{n-1} \bm{x}\left( t \right) }{d t^{n-1}} = \bm{y}_n\left( t \right) \\
    \frac{d \bm{y_n}\left( t \right) }{dt} &= \frac{d^{n} \bm{x}\left( t \right) }{d t^{n}} = \mathcal{N}'\left( t, \bm{x}\left( t \right), \frac{d \bm{x}(t)}{d t}, \ldots,\frac{d^{n-1}\bm{x}(t)}{d t^{n-1}} \right) = \mathcal{N}'\left( t, \bm{y}_1\left( t \right), \ldots, \bm{y}_n\left( t \right) \right)
.\end{align*}
Then, for $m=n m'$, define $\bm{y}:\R\to \R^{m}$ such that \[
\bm{y}\left( t \right)  = \begin{bmatrix} 
\bm{y}_1\left( t \right) \\ \vdots \\ \bm{y}_n\left( t \right) 
\end{bmatrix} 
,\] and $\mathcal{N}:\R\times \R^{m}\to \R^{m}$ such that \[
    \mathcal{N}'\left( t,\bm{y}\left( t \right)  \right) = \begin{bmatrix} 
    \bm{y}_2 \\ \vdots \\ \bm{y}_n \\ \mathcal{N}'\left( t, \bm{y}_1\left( t \right), \ldots, \bm{y}_n\left( t \right) \right)
    \end{bmatrix} 
.\] With this, it is easy to see that the first-order \gls{ODE}
\begin{equation}\label{eq:ode}
\frac{d \bm{y}\left( t \right) }{d t} = \mathcal{N}\left( t, \bm{y}\left( t \right)  \right) 
\end{equation}
is equivalent to the original $n$-th order \gls{ODE}, that is, given a solution for one of the equations, one can trivially derive the solution for the other.
This allows us to focus on first-order \gls{ODE}s for the remaining of the chapter.

\section{Initial Value Problems}

It is a common problem to find an explicit definition of the unknown functions in a differential equation.
For \gls{ODE}s, it is often the case that, if one knowns side conditions on the unknown function, than a solution exists and it is unique.
More precisely, for an \gls{ODE} defined as in \eqref{eq:ode}, conditions of the form \[
\bm{y}\left( t_0 \right) =\bm{y}_0
,\] in which $t_0\in I\subset \R$ and $\bm{y}_0\in R\subset \R^m$, $R$ a rectangle, guarantee the existence and uniqueness of a (analytic) solution in $I$ if $\mathcal{N}$ is an analytic function\footnotemark in $I\times R$, which is the case for many practical applications.[REFTO Iserles, 2008]
\footnotetext{A function is said analytic if and only if its Taylor series expansion converges in the entirety of its domain.}
The problem of finding the solution given conditions as above is called the \gls{IVP}.

\gls{IVP} shows up frequently when the current (or past) state of a system is known and the future state is desired.
Recalling Newton's second law example, suppose that, besides $F(x)=C$, it is also known that the object lies still at $t=0$, i.e., \[
y(0) = 0,\,\frac{d y(0)}{dt}=0
,\] and we are interested in the solution in the interval $I=\left[ 0,T \right] $, for $T>0$. Then, the only solution is that with $a_0=a_1=0$, that is, \[
    \phi\left( t \right) = \frac{C}{2m}t^2
.\] 

\section{Numerical Solvers}

\subsection{Euler's Method}

In an \gls{IVP}, one effectively has the value of $\bm{y}\left( t \right) $ at a given instant in time $t_0$ and can find the slope at that time through $\mathcal{N}$.
Therefore, it is natural that an efficient numerical solution is derived from the linear approximation of $\bm{y}\left( t \right) $ at $t_0$.
    In practice, this means to approximate $\mathcal{N}\left( t, \bm{y}\left( t \right) \right) \approx \mathcal{N}\left( t_0, \bm{y}_0 \right) $ in a neighborhood of $t_0$, i.e., an interval $I^h \subset I$ with length $h$ such that $t_0\in I^h$.
Then, the approximate solution is
\begin{align*}
    \bm{\phi}\left( t \right) &= \bm{y}\left( t_0 \right) + \int_{t_0}^{t}\mathcal{N}\left( \tau, \bm{y}\left( \tau \right) \right) \\
    &\approx \bm{y}\left( t_0 \right) + \left( t-t_0 \right)\mathcal{N}\left( t_0, \bm{y}_0 \right),\,t\in I^h
.\end{align*}

Of course, the performance of this naïve approach (for any non-constant $\mathcal{N}$) is satisfactory only for small $h$.
However, one can perform many of such approximations with arbitrarily small $h$.
Let us assume that $I=\left[ t_0,t_N \right]$ and one desires an approximation of an exact solution $\bm{\phi}\left( t \right) $ on $I$.
Given a sequence $t_0,\ldots,t_i,\ldots,t_N$ such that $t_{i+1}-t_i=h,\,i=1,\ldots,N$, let us call $\bm{y}_{i}$ the approximation of an exact solution at $t_i$ for $i=1,\ldots,N$.
One can use the first order approximation recursively to compute \[
\bm{y}_{i+1} = \bm{y}_{i} + h\mathcal{N}\left( t_i, \bm{y}_i \right),\,i=1,\ldots,N
\] with, in theory, arbitrary precision, once $h$, in here called the \emph{time step}, can be made arbitrarily small.[REFTO Iserles, 2008]
This approach is the \emph{Euler's method}, and forms the basis of most of the commonly used advanced numerical solvers for \gls{ODE}s.

\section{Van der Pol Oscillator}

