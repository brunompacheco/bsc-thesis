% ----------------------------------------------------------
\chapter{Introduction}\label{ch:intro}
% ----------------------------------------------------------

% \textbf{Instruções do padrão genérico de TCCs da BU:}

% As orientações aqui apresentadas são baseadas em um conjunto de normas elaboradas pela \gls{ABNT}. Além das normas técnicas, a Biblioteca também elaborou uma série de tutoriais, guias, \textit{templates} os quais estão disponíveis em seu site, no endereço \url{http://portal.bu.ufsc.br/normalizacao/}.

% Paralelamente ao uso deste \textit{template} recomenda-se que seja utilizado o \textbf{Tutorial de Trabalhos Acadêmicos} (disponível neste link \url{https://repositorio.ufsc.br/handle/123456789/180829}).

% Este \textit{template} está configurado apenas para a impressão utilizando o anverso das folhas, caso você queira imprimir usando a frente e o verso, acrescente a opção \textit{openright} e mude de \textit{oneside} para \textit{twoside} nas configurações da classe \textit{abntex2} no início do arquivo principal \textit{main.tex} \cite{abntex2classe}.

% Conforme a \href{https://repositorio.ufsc.br/bitstream/handle/123456789/197121/RN46.2019.pdf?sequence=1&isAllowed=y}{Resolução NORMATIVA nº 46/2019/CPG} as dissertações e teses não serão mais entregues em formato impresso na Biblioteca Universitária. Consulte o Repositório Institucional da UFSC ou sua Secretaria de Pós Graduação sobre os procedimentos para a entrega. 

% \nocite{NBR6023:2002}
% \nocite{NBR6027:2012}
% \nocite{NBR6028:2003}
% \nocite{NBR10520:2002}

The numerous successful applications of deep learning in the most varied industries \cite{ciregan_multi-column_2012,krizhevsky_imagenet_2012,silver_mastering_2016} have established it as a cornerstone of artificial intelligence.
With such a fame, one could easily suppose that it can be a powerful tool to solve a very common problem in the study of dynamical systems: to solve differential equations.
% The ability to learn from experience and to interpret the world through stacks of concepts on top of simpler concepts \cite{goodfellow_deep_2016} is what makes 

Yet, deep learning models gather knowledge by learning from experience, which gives it a data-driven nature.
Unfortunately, gathering enough data (samples, records, experiment history) from physical phenomena of dynamical systems can be expensive, as it may require an assortment of measurement devices and expensive material to stimulate and observe the phenomena of interest.
This makes training deep learning models in such context challenging.

\textcite{Raissi2019} proposed to optimize the dynamics of the model instead of its outputs.
This way, one can train a deep learning model to approximate the dynamics of a system, given that an accurate description of these dynamics is known (e.g., in the form of differential equations).
Therefore, only minimal data is necessary, covering the initial and boundary conditions which are usually known beforehand.

At the same time, \textcite{Bai2019} and \textcite{Ghaoui2019} proposed a novel approach that is to define the output of a model implicitly, in the form of a solution to an equilibrium equation.
Given a function $f:\R^{n+m}\to\R^m$, the output $z = F(x)$ of a so-called \gls{DEQ} $N:\R^n\to \R^m$, given an input $x\in\R^n$, is defined implicitly as the solution of $z = f(x,z)$.
A model defined this way is effectively an infinite-depth network \cite{Bai2019} with residual connections \cite{he_deep_2016}.
As to generate the output of an equilibrium model one has to solve an equilibrium equation, these models present an iterative behavior, hence similar to commonly used differential equation solvers such as the Runge-Kutta algorithm.

Even though successful applications of \gls{DEQ}s have already appeared in the literature \cite{bai_multiscale_2020}, it is still unknown, for the majority of deep learning applications, whether \gls{DEQ}s can overcome traditional models. To this end, this thesis explores a novel application: \gls{DEQ}s as \gls{ODE} solvers.

- DEQs are interesting because having less parameters => easier causality analysis, more explainability, more generalization => more robustness
    - themes that are of extremely important for practical applications in the automation industry
    - grab some references from Daniela Rus' keynote?

- Practical applications in control and automation => expensive to acquire data => need to be more data efficient

- The question this work aims to answer is: can we train DEQs in a Physiscs-Informed approach?

% \textbf{Instruções da Coordenação do PFC:}

% Neste primeiro capítulo é muito importante deixar bem claro (de uma maneira mais resumida, sem entrar em detalhes técnicos, apenas para passar a ideia geral ao leitor):

% \begin{itemize}
%     \item o problema tratado no PFC;
%     \item a importância do problema para a empresa ou instituição em que o PFC foi realizado;
%     \item a solução proposta;
%     \item a metodologia utilizada;
%     \item os resultados obtidos e a sua importância para a empresa/clientes;
%     \item o que de fato foi feito pelo autor, diferenciando do que foi aproveitado de trabalhos anteriores/outras equipes da empresa. \textbf{Importante}: esta preocupação em diferenciar o trabalho do autor daquele de possíveis colegas em um trabalho em equipe deve permear todo o documento.
% \end{itemize}

% Apesar de o tamanho da monografia não ter uma correlação direta com a nota, bons trabalhos não costumam ser relatados suficientemente bem com menos de 50 páginas. Acima de 100 páginas a monografia pode se tornar ``massante'', discorrendo além do necessário para o entendimento do trabalho, portanto perdendo o foco do leitor.

% A linguagem a ser utilizada em um trabalho acadêmico deve ser técnico-científica, portanto formal (e não informal, como se o trabalho estivesse sendo explicado a um colega ou familiar). Portanto não devem ser usadas gírias. Utilize correto ortográfico e verifique a gramática (pontuação, uso da vírgula, concordância, coesão textual etc.).

% ----------------------------------------------------------
\section{Objectives}
% ----------------------------------------------------------

This work aims to study, implement and validate the use of Deep Equilibrium models as effective and efficient \gls{ODE} solvers.
An efficient model is twofold: it does not require costly (many samples or hard-to-acquire samples) data for training, and it does not require expensive hardware to generate fast results.
At the same time, an effective model is one that can provide accurate results on a large domain, i.e., it can solve a given \gls{ODE} for a long range of its independent variable.

More specifically, the work described in here aims to:
\begin{enumerate}
    \item Implement a Deep Equilibrium model following the design proposed by \textcite{Bai2019};
    
    \item Design and implement a \gls{PINN} training algorithm suitable for \gls{DEQ}s, building on top of the work by \textcite{Raissi2019};
    \item Evaluate the physics-informed deep equilibrium model in comparison to traditional \gls{PINN}s in the task of solving an \gls{ODE}.
\end{enumerate}
With these steps, it is expected that a solid conclusion will be achieved on the suitability of \gls{DEQ}s, a state-of-the-art deep learning architecture, on the task of solving \gls{ODE}s.

% Having these goals in mind, a model implementation following the approach proposed by \textcite{Bai2019} is desired, with the allied training algorithm proposed by \textcite{Raissi2019} so the dynamics of the model is optimized to match the dynamics of a target system.


% Aqui são descritos os objetivos, que podem ser estratificados em objetivo geral e objetivos específicos.

% Outra opção é colocar os objetivos específicos na forma de passos de uma metodologia de trabalho, ou seja, os procedimentos e ferramentas adotadas em cada fase do projeto (um plano de trabalho).

\section*{}

\textcolor{red}{- Estrutura do documento}
