% Modelo UFSC/CTC/DAS para PFCs (08/07/2020)
% Autores: Matheus Bruhns Bastos e Marcelo De Lellis Costa de Oliveira
% --------------------------------------------------------
% Adaptado do arquivo Template-Trabalhos-Academicos-UFSC-A4-v1.3
% disponibilizado em http://portal.bu.ufsc.br/files/2013/10/Template-Trabalhos-Academicos-UFSC-A4-v1.3.zip
% Modelo UFSC para Trabalhos Academicos (tese de doutorado, dissertação de
% mestrado) utilizando a classe abntex2
%
% Autor: Alisson Lopes Furlani
% 	Modificações:
%	- 27/08/2019: Alisson L. Furlani, add 'glossaries' package
%   - 30/10/2019: Alisson L. Furlani, adjusted some spacing errors and changed math fonts
%   - 17/01/2020: Alisson L. Furlani, updated certification page
%   - 07/02/2020: Alisson L. Furlani, fixed table counter bug
%   - 11/03/2020: Alisson L. Furlani, changed greek letters in math and fixed citation style
% ------------------------------------------------------------------------
% ------------------------------------------------------------------------

\documentclass[
	% -- opções da classe memoir --
	12pt,				% tamanho da fonte
	%openright,			% capítulos começam em pág ímpar (insere página vazia caso preciso)
	oneside,			% para impressão no anverso. Oposto a twoside
	a4paper,			% tamanho do papel. 
	% -- opções da classe abntex2 --
	chapter=TITLE,		% títulos de capítulos convertidos em letras maiúsculas
	section=TITLE,		% títulos de seções convertidos em letras maiúsculas
	%subsection=TITLE,	% títulos de subseções convertidos em letras maiúsculas
	%subsubsection=TITLE,% títulos de subsubseções convertidos em letras maiúsculas
	% -- opções do pacote babel --
	brazil,			% idioma adicional para hifenização
	%french,				% idioma adicional para hifenização
	%spanish,			% idioma adicional para hifenização
	english				% o último idioma é o principal do documento
	]{abntex2}

\usepackage{setup/ufscthesisA4-alf}
\addbibresource{aftertext/references.bib} % Seus arquivos de referências

% ---
% Filtering and Mapping Bibliographies
% ---
\DeclareSourcemap{
	\maps[datatype=bibtex]{
		% remove fields that are always useless
		\map{
			\step[fieldset=abstract, null]
			\step[fieldset=pagetotal, null]
		}
		% remove URLs for types that are primarily printed
%		\map{
%			\pernottype{software}
%			\pernottype{online}
%			\pernottype{report}
%			\pernottype{techreport}
%			\pernottype{standard}
%			\pernottype{manual}
%			\pernottype{misc}
%			\step[fieldset=url, null]
%			\step[fieldset=urldate, null]
%		}
		\map{
			\pertype{inproceedings}
			% remove mostly redundant conference information
			\step[fieldset=venue, null]
			\step[fieldset=eventdate, null]
			\step[fieldset=eventtitle, null]
			% do not show ISBN for proceedings
			\step[fieldset=isbn, null]
			% Citavi bug
			\step[fieldset=volume, null]
		}
	}
}
% ---

% ---
% Informações de dados para CAPA e FOLHA DE ROSTO
% ---
\autor{Bruno Machado Pacheco}
% Substituir 'Título do trabalho' pelo título da trabalho.
\titulo{Physics-Informed Deep Equilibrium Models for Solving ODEs}
% Caso não tenha substítulo, comente a linha a seguir.
% \subtitulo{subtitle (if any)}
% Substituir 'XXXXXX' pelo nome do seu
% orientador.
\orientador{Prof. Eduardo Camponogara, Ph.D.}
% Se for orientado por uma mulher, comente a linha acima e descomente a linha a seguir.
% \orientador[Orientadora]{Nome da orientadora, Dra.}
% Substituir 'XXXXXX' pelo nome do seu
% supervisor no local de realização do PFC. Caso não tenha supervisor, comente a linha a seguir.
% \coorientador{YYYYY, Eng.}
% Se for supervisionado por uma mulher, comente a linha acima e descomente a linha a seguir.
% \coorientador[Supervisora]{XXXXXX, Eng.}
% Substituir '[ano]' pelo ano (ano) em que seu trabalho foi defendido.
\ano{2022}
% FIXME Substituir '[dia] de [mês] de [ano]' pela data em que ocorreu sua defesa.
%\data{[dia] de [mês] de [ano]}
% Substituir 'Local' pela cidade em que ocorreu sua defesa.
\local{Florianópolis}
\instituicaosigla{UFSC}
\instituicao{Federal University of Santa Catarina}
% Substituir 'Dissertação/Tese' pelo tipo de trabalho (Tese, Dissertação). 
\tipotrabalho{Final report of the subject DAS5511 (Course Final Project) as a Concluding Dissertation}
%Relatório final da disciplina DAS5511 (Projeto de Fim de Curso) como Trabalho de Conclusão}
\formacao{Control and Automation Engineer}
% Substituir '[mestrado/doutorado]' pelo nivel adequado.
\nivel{[mestrado/doutorado]}
% Substituir 'Programa de Pós-Graduação em XXXXXX' pela curso adequado.
\programa{Programa de Pós-Graduação em XXXXXX}
\centro{Technology Center}
\departamento{Department of Automation and Systems Engineering}
\curso{Undergraduate Program in Control and Automation Engineering}
\preambulo
{%
%\imprimirtipotrabalho~do~\imprimircurso~da~\imprimirinstituicao~como~requisito~para~a~obtenção~do~título~de~\imprimirformacao.
\imprimirtipotrabalho~of the~\imprimircurso~of the~\imprimirinstituicao.
}
% ---

% ---
% Configurações de aparência do PDF final
% ---
% alterando o aspecto da cor azul
\definecolor{blue}{RGB}{41,5,195}
% informações do PDF
\makeatletter
\hypersetup{
     	%pagebackref=true,
		pdftitle={\@title}, 
		pdfauthor={\@author},
    	pdfsubject={\imprimirpreambulo},
	    pdfcreator={LaTeX with abnTeX2},
		pdfkeywords={ufsc, latex, abntex2}, 
		colorlinks=true,       		% false: boxed links; true: colored links
    	linkcolor=black,%blue,          	% color of internal links
    	citecolor=black,%blue,        		% color of links to bibliography
    	filecolor=black,%magenta,      		% color of file links
		urlcolor=black,%blue,
		bookmarksdepth=4
}
\makeatother
% ---

% ---
% compila a lista de abreviaturas e siglas e a lista de símbolos
% ---

% Declaração das siglas
\siglalista{ABNT}{Associação Brasileira de Normas Técnicas}
\siglalista{ODE}{ordinary differential equation}
\siglalista{PINN}{physics informed neural network}
\siglalista{DEQ}{deep equilibrium model}
\siglalista{IVP}{initial-value problem}
\siglalista{RK4}{classic Runge-Kutta method}

% Declaração dos simbolos
\simbololista{lr}{\ensuremath{\gamma}}{Learning rate}
\simbololista{param}{\ensuremath{\theta}}{Parameter vector}
\simbololista{sigmoid}{\ensuremath{\sigma}}{Sigmoid function}
% \simbololista{r}{\ensuremath{r}}{Raio de um círculo}
% \simbololista{A}{\ensuremath{A}}{Área de um círculo}

% compila a lista de abreviaturas e siglas e a lista de símbolos
\makenoidxglossaries 

% ---

% ---
% compila o indice
% ---
\makeindex
% ---

% ----
% Início do documento
% ----
\begin{document}

% Seleciona o idioma do documento (conforme pacotes do babel)
\selectlanguage{english}

% Retira espaço extra obsoleto entre as frases.
\frenchspacing 

% Espaçamento 1.5 entre linhas
\OnehalfSpacing

% Corrige justificação
%\sloppy

% ----------------------------------------------------------
% ELEMENTOS PRÉ-TEXTUAIS
% ----------------------------------------------------------
% \pretextual %a macro \pretextual é acionado automaticamente no início de \begin{document}
% ---
% Capa, folha de rosto, ficha bibliografica, errata, folha de apróvação
% Dedicatória, agradecimentos, epígrafe, resumos, listas
% ---
% ---
% Capa
% ---
\imprimircapa
% ---

% ---
% Folha de rosto
% (o * indica que haverá a ficha bibliográfica)
% ---
\imprimirfolhaderosto*
% ---

% ---
% Inserir a ficha bibliografica
% ---
% http://ficha.bu.ufsc.br/
\begin{fichacatalografica}
	\includepdf{beforetext/Ficha_Catalografica.pdf}
\end{fichacatalografica}
% ---

% ---
% Inserir folha de aprovação
% ---
\begin{folhadeaprovacao}
	\OnehalfSpacing
	\centering
	\imprimirautor\\%
	\vspace*{10pt}		
	\textbf{\imprimirtitulo}%
	\ifnotempty{\imprimirsubtitulo}{:~\imprimirsubtitulo}\\%
	%		\vspace*{31.5pt}%3\baselineskip
	\vspace*{\baselineskip}
	
	This dissertation was evaluated in the context of the subject DAS5511 (Course Final Project) and approved in its final form by the \imprimircurso\\
	%Esta monografia foi julgada no contexto da disciplina DAS5511 (Projeto de Fim de Curso) e aprovada em sua forma final pelo \imprimircurso\\
	\vspace*{\baselineskip}
	Florianópolis, <month> <day>, <year>.\\
	
	%%%%%%%%%%%%%%%%%%%%%%%%%%%%%%%%%%%%%%%%%%%%%
	%IMPORTANT: no signatures are required below!
	%%%%%%%%%%%%%%%%%%%%%%%%%%%%%%%%%%%%%%%%%%%%%
	
	\vspace*{2\baselineskip}
	%\rule{0.4\textwidth}{0.4pt}\\
	Prof. xxxx, Dr.\\
	Course Coordinator\\
	
	\vspace*{\baselineskip}
	\textbf{Examining Board:} \\
	
	
	\vspace*{2\baselineskip}
	%\rule{0.4\textwidth}{0.4pt}\\
	Prof. xxxx, Dr.\\
	Advisor \\
	UFSC/CTC/DAS\\
	
	\vspace*{2\baselineskip}
	%\rule{0.4\textwidth}{0.4pt}\\
	xxxx, Eng.\\
	Supervisor \\
	Company/University xxxx\\
	
	\vspace*{2\baselineskip}
	%\rule{0.4\textwidth}{0.4pt}\\
	Prof. xxxx, Dr.\\
	Evaluator \\
	Instituição xxxx\\
	
	\vspace*{2\baselineskip}
	%\rule{0.4\textwidth}{0.4pt}\\
	Prof. xxxx, Dr.\\
	Board President \\
	UFSC/CTC/DAS
\end{folhadeaprovacao}
% ---

% ---
% Dedicatória
% ---
\begin{dedicatoria}
	\vspace*{\fill}
	\noindent
	\begin{adjustwidth*}{}{5.5cm} 
		\raggedleft       
		This work is dedicated to my classmates and my dear parents.
	\end{adjustwidth*}
\end{dedicatoria}
% ---

% ---
% Agradecimentos
% ---
\begin{agradecimentos}
	Inserir os agradecimentos aos colaboradores à execução do trabalho. 
	
	Xxxxxxxxxxxxxxxxxxxxxxxxxxxxxxxxxxxxxxxxxxxxxxxxxxxxxxxxxxxxxxxxxxxxxx. 
\end{agradecimentos}
% ---

% ---
% Epígrafe
% ---
\begin{epigrafe}
	\vspace*{\fill}
	\begin{flushright}
		\textit{``The sciences do not try to explain, they hardly even try to interpret, they mainly make models. By a model is meant a mathematical construct which, with the addition of certain verbal interpretations, describes observed phenomena. The justification of such a mathematical construct is solely and precisely that it is expected to work - that is correctly to describe phenomena from a reasonably wide area. Furthermore, it must satisfy certain esthetic criteria - that is, in relation to how much it describes, it must be rather simple.''\\
			(VON NEUMANN, 1955)}% \cite{leary_unity_1955}
	\end{flushright}
\end{epigrafe}
% ---

% ---
% RESUMOS
% ---

% resumo em português
\setlength{\absparsep}{18pt} % ajusta o espaçamento dos parágrafos do resumo
\begin{resumo}
	\SingleSpacing
	\Glspl{PINN} and \glspl{DEQ} are novel approaches that approximate deep learning's representational power to applications with realistic requirements, such as robustness, data scarcity and explainability.
	The former proposes an efficient way to train neural networks to model physical phenomena.
	The latter is a new model architecture that can provide more representational power with fewer parameters.
	This work aims to study both and apply them to solve \glspl{IVP} of \glspl{ODE}, in an approach named \gls{PIDEQ}.
	We implement the proposed approach and test it, analysing the impacts of the multiple hyperparameters in the approximate solution.
	Our results show that indeed \gls{PIDEQ} models are able to solve \glspl{IVP}, providing approximate solutions with small errors.
% 	\textbf{Instruções do padrão genérico de TCCs da BU:}
	% No resumo são ressaltados o objetivo da pesquisa, o método utilizado, as discussões e os resultados com destaque apenas para os pontos principais. O resumo deve ser significativo, composto de uma sequência de frases concisas, afirmativas, e não de uma enumeração de tópicos. Não deve conter citações. Deve usar o verbo na voz ativa e na terceira pessoa do singular. O texto do resumo deve ser digitado, em um único bloco, sem espaço de parágrafo. O espaçamento entre linhas é simples e o tamanho da fonte é 12. Abaixo do resumo, informar as palavras-chave (palavras ou expressões significativas retiradas do texto) ou, termos retirados de thesaurus da área. Deve conter de 150 a 500 palavras. O resumo é elaborado de acordo com a NBR 6028. \textbf{Instruções da Coordenação do PFC:} o resumo deve contextualizar o trabalho e descrever de forma sucinta o problema, a solução proposta, a metodologia utilizada e os resultados obtidos. Tudo de forma bem resumida e direta, sem entrar em detalhes técnicos.
	
	\textbf{Keywords}: Physics-Informed Neural Network; Deep Equilibrium Model; Implict Model; Initial-Value Problem; Deep Learning.
\end{resumo}

% resumo em inglês
\begin{resumo}[Resumo]
	\SingleSpacing
	\begin{otherlanguage*}{brazil}
	    Redes neurais para solução de problemas físicos, denominadas \emph{physics-informed neural networks} (PINNs), e modelos profundos de equilíbrio (do inglês, DEQs), são contribuições recentes que facilitam o uso de modelos de aprendizagem profunda, conhecidos pela capacidade representativa, de aplicações com requisitos realistas de robustez, explicabilidade e escassez de dados.
	    PINNs se mostraram uma forma eficiente de treinar redes neurais para modelar fenômenos físicos.
	    DEQ, por outro lado, é uma nova arquitetura que promete mais capacidade de representação com menos parâmetros.
	    Este trabalho consiste em um estudo de ambos, além de uma aplicação que combina-os para resolver problemas de valor inicial de equações diferenciais ordinárias, com um modelo chamado PIDEQ.
	    A abordagem proposta para resolver esse tipo de problema foi implementada e testada utilizando o oscilador de Van der Pol, com uma análise de impacto dos seus diferentes hiperparâetros.
	    Os resultados mostram que, de fato, é possível treinar um PIDEQ para resolver o problema proposto, gerando soluções aproximadas com baixo erro.
		
	    \textbf{Keywords}: Physics-Informed Neural Network; Deep Equilibrium Model; Implict Model; Initial-Value Problem; Deep Learning.
	    \textbf{Palavras-chave}: Palavra-chave 1. Palavra-chave 2. Palavra-chave 3.
	\end{otherlanguage*}
\end{resumo}

%% resumo em francês 
%\begin{resumo}[Résumé]
% \begin{otherlanguage*}{french}
%    Il s'agit d'un résumé en français.
% 
%   \textbf{Mots-clés}: latex. abntex. publication de textes.
% \end{otherlanguage*}
%\end{resumo}
%
%% resumo em espanhol
%\begin{resumo}[Resumen]
% \begin{otherlanguage*}{spanish}
%   Este es el resumen en español.
%  
%   \textbf{Palabras clave}: latex. abntex. publicación de textos.
% \end{otherlanguage*}
%\end{resumo}
%% ---

{%hidelinks
	\hypersetup{hidelinks}
	% ---
	% inserir lista de ilustrações
	% ---
	\pdfbookmark[0]{\listfigurename}{lof}
	\listoffigures*
	\cleardoublepage
	% ---
	
	% ---
	% inserir lista de quadros
	% ---
	\pdfbookmark[0]{\listofquadrosname}{loq}
	\listofquadros*
	\cleardoublepage
	% ---
	
	% ---
	% inserir lista de tabelas
	% ---
	\pdfbookmark[0]{\listtablename}{lot}
	\listoftables*
	\cleardoublepage
	% ---
	
	% ---
	% inserir lista de abreviaturas e siglas (devem ser declarados no preambulo)
	% ---
	\imprimirlistadesiglas
	% ---
	
	% ---
	% inserir lista de símbolos (devem ser declarados no preambulo)
	% ---
	\imprimirlistadesimbolos
	% ---
	
	% ---
	% inserir o sumario
	% ---
	\pdfbookmark[0]{\contentsname}{toc}
	\tableofcontents*
	\cleardoublepage
	
}%hidelinks
% ---

% ---

% ----------------------------------------------------------
% ELEMENTOS TEXTUAIS
% ----------------------------------------------------------
\textual

% ---
% 1 - Introdução
% ---
% ----------------------------------------------------------
\chapter{Introduction}\label{ch:intro}
% ----------------------------------------------------------

% \textbf{Instruções do padrão genérico de TCCs da BU:}

% As orientações aqui apresentadas são baseadas em um conjunto de normas elaboradas pela \gls{ABNT}. Além das normas técnicas, a Biblioteca também elaborou uma série de tutoriais, guias, \textit{templates} os quais estão disponíveis em seu site, no endereço \url{http://portal.bu.ufsc.br/normalizacao/}.

% Paralelamente ao uso deste \textit{template} recomenda-se que seja utilizado o \textbf{Tutorial de Trabalhos Acadêmicos} (disponível neste link \url{https://repositorio.ufsc.br/handle/123456789/180829}).

% Este \textit{template} está configurado apenas para a impressão utilizando o anverso das folhas, caso você queira imprimir usando a frente e o verso, acrescente a opção \textit{openright} e mude de \textit{oneside} para \textit{twoside} nas configurações da classe \textit{abntex2} no início do arquivo principal \textit{main.tex} \cite{abntex2classe}.

% Conforme a \href{https://repositorio.ufsc.br/bitstream/handle/123456789/197121/RN46.2019.pdf?sequence=1&isAllowed=y}{Resolução NORMATIVA nº 46/2019/CPG} as dissertações e teses não serão mais entregues em formato impresso na Biblioteca Universitária. Consulte o Repositório Institucional da UFSC ou sua Secretaria de Pós Graduação sobre os procedimentos para a entrega. 

% \nocite{NBR6023:2002}
% \nocite{NBR6027:2012}
% \nocite{NBR6028:2003}
% \nocite{NBR10520:2002}

The numerous successful applications of deep learning in the most varied industries \cite{ciregan_multi-column_2012,krizhevsky_imagenet_2012,silver_mastering_2016} have established it as a cornerstone of artificial intelligence.
With such a fame, one could easily suppose that it can be a powerful tool to solve a very common problem in the study of dynamical systems: to solve differential equations.
% The ability to learn from experience and to interpret the world through stacks of concepts on top of simpler concepts \cite{goodfellow_deep_2016} is what makes 

Yet, deep learning models gather knowledge by learning from experience, which gives it a data-driven nature.
Unfortunately, gathering enough data (samples, records, experiment history) from physical phenomena of dynamical systems can be expensive, as it may require an assortment of measurement devices and expensive material to stimulate and observe the phenomena of interest.
This makes training deep learning models in such context challenging.

\textcite{Raissi2019} proposed to optimize the dynamics of the model instead of its outputs.
This way, one can train a deep learning model to approximate the dynamics of a system, given that an accurate description of these dynamics is known (e.g., in the form of differential equations).
Therefore, only minimal data is necessary, covering the initial and boundary conditions which are usually known beforehand.

At the same time, \textcite{Bai2019} and \textcite{Ghaoui2019} proposed a novel approach that is to define the output of a model implicitly, in the form of a solution to an equilibrium equation.
Given a function $f:\R^{n+m}\to\R^m$, the output $z = F(x)$ of a so-called \gls{DEQ} $N:\R^n\to \R^m$, given an input $x\in\R^n$, is defined implicitly as the solution of $z = f(x,z)$.
A model defined this way is effectively an infinite-depth network \cite{Bai2019} with residual connections \cite{he_deep_2016}.
As to generate the output of an equilibrium model one has to solve an equilibrium equation, these models present an iterative behavior, hence similar to commonly used differential equation solvers such as the Runge-Kutta algorithm.

Even though successful applications of \gls{DEQ}s have already appeared in the literature \cite{bai_multiscale_2020}, it is still unknown, for the majority of deep learning applications, whether \gls{DEQ}s can overcome traditional models. To this end, this thesis explores a novel application: \gls{DEQ}s as \gls{ODE} solvers.

- DEQs are interesting because having less parameters => easier causality analysis, more explainability, more generalization => more robustness
    - themes that are of extremely important for practical applications in the automation industry
    - grab some references from Daniela Rus' keynote?

- Practical applications in control and automation => expensive to acquire data => need to be more data efficient

- The question this work aims to answer is: can we train DEQs in a Physiscs-Informed approach?

% \textbf{Instruções da Coordenação do PFC:}

% Neste primeiro capítulo é muito importante deixar bem claro (de uma maneira mais resumida, sem entrar em detalhes técnicos, apenas para passar a ideia geral ao leitor):

% \begin{itemize}
%     \item o problema tratado no PFC;
%     \item a importância do problema para a empresa ou instituição em que o PFC foi realizado;
%     \item a solução proposta;
%     \item a metodologia utilizada;
%     \item os resultados obtidos e a sua importância para a empresa/clientes;
%     \item o que de fato foi feito pelo autor, diferenciando do que foi aproveitado de trabalhos anteriores/outras equipes da empresa. \textbf{Importante}: esta preocupação em diferenciar o trabalho do autor daquele de possíveis colegas em um trabalho em equipe deve permear todo o documento.
% \end{itemize}

% Apesar de o tamanho da monografia não ter uma correlação direta com a nota, bons trabalhos não costumam ser relatados suficientemente bem com menos de 50 páginas. Acima de 100 páginas a monografia pode se tornar ``massante'', discorrendo além do necessário para o entendimento do trabalho, portanto perdendo o foco do leitor.

% A linguagem a ser utilizada em um trabalho acadêmico deve ser técnico-científica, portanto formal (e não informal, como se o trabalho estivesse sendo explicado a um colega ou familiar). Portanto não devem ser usadas gírias. Utilize correto ortográfico e verifique a gramática (pontuação, uso da vírgula, concordância, coesão textual etc.).

% ----------------------------------------------------------
\section{Objectives}
% ----------------------------------------------------------

This work aims to study, implement and validate the use of Deep Equilibrium models as effective and efficient \gls{ODE} solvers.
An efficient model is twofold: it does not require costly (many samples or hard-to-acquire samples) data for training, and it does not require expensive hardware to generate fast results.
At the same time, an effective model is one that can provide accurate results on a large domain, i.e., it can solve a given \gls{ODE} for a long range of its independent variable.

More specifically, the work described in here aims to:
\begin{enumerate}
    \item Implement a Deep Equilibrium model following the design proposed by \textcite{Bai2019};
    
    \item Design and implement a \gls{PINN} training algorithm suitable for \gls{DEQ}s, building on top of the work by \textcite{Raissi2019};
    \item Evaluate the physics-informed deep equilibrium model in comparison to traditional \gls{PINN}s in the task of solving an \gls{ODE}.
\end{enumerate}
With these steps, it is expected that a solid conclusion will be achieved on the suitability of \gls{DEQ}s, a state-of-the-art deep learning architecture, on the task of solving \gls{ODE}s.

% Having these goals in mind, a model implementation following the approach proposed by \textcite{Bai2019} is desired, with the allied training algorithm proposed by \textcite{Raissi2019} so the dynamics of the model is optimized to match the dynamics of a target system.


% Aqui são descritos os objetivos, que podem ser estratificados em objetivo geral e objetivos específicos.

% Outra opção é colocar os objetivos específicos na forma de passos de uma metodologia de trabalho, ou seja, os procedimentos e ferramentas adotadas em cada fase do projeto (um plano de trabalho).

\section*{}

\textcolor{red}{- Estrutura do documento}

% ---

% ---
% 2 - ODEs
% ---
% ----------------------------------------------------------
\chapter{Ordinary Differential Equations}
% ----------------------------------------------------------

This first chapter is a short review of \gls{ODE}s and how to solve them numerically.
It would be unreasonable to assume that one chapter of a bachelor thesis could cover what many authors have dedicated entire books to communicate.
Therefore, it is assumed that the reader has a basic (undergraduate level) understanding of the theory of differential equations\footnotemark.
\footnotetext{Otherwise, the books of \textcite{zill_first_2013} and \textcite{simmons_differential_2017} are great resources.}
This way, this chapter focuses on reviewing the fundamentals and establishing a notation concise for the following chapters, besides introducing the approach to numerical solve \gls{ODE}s.

\section{Introduction and Definitions}

A differential equation can be seen as a description of the relationship between unknown quantities and their rates of changes.
Because of this broader definition and direct relationship to applications, differential equations arise naturally in many fields of applied sciences\cite{hairer_solving_1993}, as it is often the case that the rate of change of a certain quantity of interest is related to the rate of change of other quantities.
A classical example of a differential equation is Newton's second law of motion \[
F\left( x\left( t \right) \right) = m \frac{d^2 y(t)}{d t^2}
,\] in which $x\left( t \right) $ is the position of an object at time $t$, $m$ is its mass, and $F(x)$ is the force under which the object is at a given position.
This example highlights one of the great powers of the differential equations, which is to describe the dynamics of a certain phenomenon without explicitly defining it.


For a more tangible definition, any equation that contains the derivatives of (at least) one unknown function with respect to (at least) one independent variable is called a \emph{differential equation}\cite{zill_first_2013}.
A differential equation that involves only \emph{ordinary} derivatives, that is, only derivatives of functions with respect to a single variable, (such as the one above) is called an \gls{ODE}.
% A differential equation can also be classified through its \emph{order} (the order of its highest derivative)
\gls{ODE}s can be represented in the normal form \[
    \frac{d^n \bm{x}(t)}{d t^{n}} = \mathcal{N}'\left( t, \bm{x}\left( t \right), \frac{d \bm{x}(t)}{d t}, \ldots,\frac{d^{n-1}\bm{x}(t)}{d t^{n-1}} \right)
.\] 
in which $\bm{x}:\R\to \R^{m'} $, $\mathcal{N}':\R\times \R^{nm'}\to \R^{m'}$ is a continuous function, $\frac{d^n \bm{x}(t)}{d t^{n}}$ is the vector containing the derivatives with respect to $t$ of all $\bm{x}\left( t \right) $ components\footnotetext{I.e., the first (and only) row-vector of the Jacobian of $\bm{x}\left( t \right) $.}, and $n$ is the order of the highest derivative in the equation and is commonly referred to as the \emph{order of the differential equation}.

Any function $\bm{\phi}:I\subset\R\to \R^{m'}$ is said to be a \emph{solution} of an $n$-th order \gls{ODE} if it satisfies \[
    \frac{d^n \bm{\phi}(t)}{d t^{n}} = \mathcal{N}\left( t, \bm{\phi}\left( t \right) , \frac{d \bm{\phi}(t)}{d t}, \ldots,\frac{d^{n-1}\bm{\phi}(t)}{d t^{n-1}} \right),\,\forall t\in I
.\] Note, however, that the solutions are not necessarily unique.
As an example, given Newton's second law of motion with constant force ($F(x)=C$), then it is easy to see that any second-order polynomial of the form \[
    \phi\left( t \right) = \frac{C}{2m}t^2 + a_1t + a_0,\, a_0,a_1\in \R
\] is a solution.

Now, given an $n$-th order \gls{ODE}, let $\bm{y}_1\left( t \right) =\bm{x}\left( t \right),\,\bm{y}_2\left( t \right) = \frac{d \bm{x}\left( t \right) }{d t} ,\ldots, \bm{y}_n\left( t \right) = \frac{d^{n-1} \bm{x}\left( t \right) }{d t^{n-1}}$.
Note that we can now write the following \emph{system} of first-order differential equations
\begin{align*}
    \frac{d \bm{y_1}\left( t \right) }{dt} &= \frac{d \bm{x}\left( t \right) }{d t} = \bm{y}_2\left( t \right) \\
    &\vdots \\
    \frac{d \bm{y_{n-1}}\left( t \right) }{dt} &= \frac{d^{n-1} \bm{x}\left( t \right) }{d t^{n-1}} = \bm{y}_n\left( t \right) \\
    \frac{d \bm{y_n}\left( t \right) }{dt} &= \frac{d^{n} \bm{x}\left( t \right) }{d t^{n}} = \mathcal{N}'\left( t, \bm{x}\left( t \right), \frac{d \bm{x}(t)}{d t}, \ldots,\frac{d^{n-1}\bm{x}(t)}{d t^{n-1}} \right) = \mathcal{N}'\left( t, \bm{y}_1\left( t \right), \ldots, \bm{y}_n\left( t \right) \right)
.\end{align*}
Then, for $m=n m'$, define $\bm{y}:\R\to \R^{m}$ such that \[
\bm{y}\left( t \right)  = \begin{bmatrix} 
\bm{y}_1\left( t \right) \\ \vdots \\ \bm{y}_n\left( t \right) 
\end{bmatrix} 
,\] and $\mathcal{N}:\R\times \R^{m}\to \R^{m}$ such that \[
    \mathcal{N}'\left( t,\bm{y}\left( t \right)  \right) = \begin{bmatrix} 
    \bm{y}_2 \\ \vdots \\ \bm{y}_n \\ \mathcal{N}'\left( t, \bm{y}_1\left( t \right), \ldots, \bm{y}_n\left( t \right) \right)
    \end{bmatrix} 
.\] With this, it is easy to see that the first-order \gls{ODE}
\begin{equation}\label{eq:ode}
\frac{d \bm{y}\left( t \right) }{d t} = \mathcal{N}\left( t, \bm{y}\left( t \right)  \right) 
\end{equation}
is equivalent to the original $n$-th order \gls{ODE}, that is, given a solution for one of the equations, one can trivially derive the solution for the other.
This allows us to focus on first-order \gls{ODE}s for the remaining of the chapter.

\section{Initial Value Problems}

It is a common problem to find an explicit definition of the unknown functions in a differential equation.
For \gls{ODE}s, it is often the case that, if one knowns side conditions on the unknown function, than a solution exists and it is unique.
More precisely, for an \gls{ODE} defined as in \eqref{eq:ode}, conditions of the form \[
\bm{y}\left( t_0 \right) =\bm{y}_0
,\] in which $t_0\in I\subset \R$ and $\bm{y}_0\in R\subset \R^m$, $R$ a rectangle, guarantee the existence and uniqueness of a (analytic) solution in $I$ if $\mathcal{N}$ is an analytic function\footnotemark in $I\times R$, which is the case for many practical applications.\cite{iserles_first_2008}
\footnotetext{A function is said analytic if and only if its Taylor series expansion converges in the entirety of its domain.}
The problem of finding the solution given conditions as above is called the \gls{IVP}.

\gls{IVP} shows up frequently when the current (or past) state of a system is known and the future state is desired.
Recalling Newton's second law example, suppose that, besides $F(x)=C$, it is also known that the object lies still at $t=0$, i.e., \[
y(0) = 0,\,\frac{d y(0)}{dt}=0
,\] and we are interested in the solution in the interval $I=\left[ 0,T \right] $, for $T>0$. Then, the only solution is that with $a_0=a_1=0$, that is, \[
    \phi\left( t \right) = \frac{C}{2m}t^2
.\] 

\section{Numerical Solvers}

\subsection{Euler's Method}

In an \gls{IVP}, one effectively has the value of $\bm{y}\left( t \right) $ at a given instant in time $t_0$ and can find the slope at that time through $\mathcal{N}$.
Therefore, it is natural that an efficient numerical solution is derived from the linear approximation of $\bm{y}\left( t \right) $ at $t_0$.
    In practice, this means to approximate $\mathcal{N}\left( t, \bm{y}\left( t \right) \right) \approx \mathcal{N}\left( t_0, \bm{y}_0 \right) $ in a neighborhood of $t_0$, i.e., an interval $I^h \subset I$ with length $h$ such that $t_0\in I^h$.
Then, the approximate solution is
\begin{align*}
    \bm{\phi}\left( t \right) &= \bm{y}\left( t_0 \right) + \int_{t_0}^{t}\mathcal{N}\left( \tau, \bm{y}\left( \tau \right) \right) \\
    &\approx \bm{y}\left( t_0 \right) + \left( t-t_0 \right)\mathcal{N}\left( t_0, \bm{y}_0 \right),\,t\in I^h
.\end{align*}

Of course, the performance of this naïve approach (for any non-constant $\mathcal{N}$) is satisfactory only for small $h$.
However, one can perform many of such approximations with arbitrarily small $h$.
Let us assume that $I=\left[ t_0,t_N \right]$ and one desires an approximation of an exact solution $\bm{\phi}\left( t \right) $ on $I$.
Given a sequence $t_0,\ldots,t_i,\ldots,t_N$ such that $t_{i+1}-t_i=h,\,i=1,\ldots,N$, let us call $\bm{y}_{i}$ the approximation of an exact solution at $t_i$ for $i=1,\ldots,N$.
One can use the first order approximation recursively to compute \[
\bm{y}_{i+1} = \bm{y}_{i} + h\mathcal{N}\left( t_i, \bm{y}_i \right),\,i=1,\ldots,N
\] with, in theory, arbitrary precision, once $h$, in here called the \emph{time step}, can be made arbitrarily small.\cite{iserles_first_2008}
This approach is the \emph{Euler's method}, and forms the basis of most of the commonly used advanced numerical solvers for \gls{ODE}s.

\section{Van der Pol Oscillator}


% ---

% ---
% 3 - Deep Learning
% ---
% ----------------------------------------------------------
\chapter{Deep Learning}\label{ch:deep-learning}
% ----------------------------------------------------------

In this chapter, a brief overview of the key elements of deep learning are presented.
Deep Learning is a broad area of knowledge that can be seen (and presented) through many viewpoints, which results in a multitude of notations and few well-established naming conventions.
Here, the approach of \textcite{goodfellow_deep_2016} is followed, being altered only to make it consistent with the other chapters.
Furthermore, even though physics-informed learning and implicit models can be seen as branches of deep learning, they are introduced in separated chapters for their relevance.

% ----------------------------------------------------------
\section{Origin/Introduction/Definition?}
% ----------------------------------------------------------

Even though \textit{Deep Learning} may seem like a novel and exciting technology, it has been studied under many different names since the 1940s \cite{goodfellow_deep_2016}.
Deep learning involves composing multiple levels\footnote{The understanding of these levels as a learning \textit{depth} is the origin of the naming for deep learning.} of representation learning.
Representation learning, in turn, is about automatically extracting higher level features from the input data \cite{lecun_deep_2015,bengio_representation_2013}, aiming to facilitate the extraction of useful information, for example, to classify the data into given categories.

Following these definitions, we can see a deep learning \textit{model} as a function with multiple levels of representations.
A simple way of putting this definition into terms would be to define a model with $L$ layers as a function $f$ such that, for a given input $x$,
\begin{align*}
    z^{[0]} &= x \\
    z^{[i]} &= f^{[i]}(z^{[i-1]}),\,i=1,\ldots,L \\
    f(x) &= z^{[L]}
.\end{align*}
From this notation, it is easy to see that each function $f^{[i]}$ maps the outputs of the previous layer into the input of the next, ideally achieving a higher abstraction level.

In this context, we can see the goal of \textit{learning} as finding parameters $\theta$ such that the layers indeed extract the desired information, culminating in the model returning the desired output.
A classic toy example in deep learning models is to approximate the Exclusive-OR function, that is, to find a model $f : \R^2 \to \R$ such that
\begin{align*}
    f([0,0]^T) = f([1,1]^T) = 0 \\
    f([1,0]^T) = f([0,1]^T) = 1 \\
.\end{align*}
We can construct this model's layers as functions $f^{[1]}:\R^2\to\R^2$ and $f^{[2]}:\R^2\to\R$ such that
\begin{align*}
    f^{[1]}(\bm{z}) &= \begin{bmatrix}
    OR(\bm{z}) \\
    NAND(\bm{z})
    \end{bmatrix} \\
    f^{[2]}(\bm{z}) &= AND(\bm{z})
.\end{align*}
Note how these functions are extracting simple information from the input, yet their stacking can approximate very well the desired behavior.

\section{Deep Feedforward Networks}\label{sec:neural-nets}

Deep Feedforward Networks are the most essential deep learning models.
They were inspired by the works of \textcite{rosenblatt_perceptron_1957} and refined through the many decades since, culminating in one of the most important deep learning models for practitioners and the basis of the most prominent results seen in the recent years \cite{goodfellow_deep_2016}.

In this model, each component is a simple affine operator followed by a nonlinear \textit{activation} function, i.e., following the notation previously presented, we can define a deep feedforward neural network with $L$ layers as a \textit{parametrized} function $f_\theta$ such that, for a given input $\bm{x}$,
\begin{align*}
    \bm{z}^{[0]} &= \bm{x} \\
    \bm{z}^{[i]} &= f_{\theta^i}^{[i]}(\bm{z}^{[i-1]}) = g^{[i]}\left(A^{[i]}\bm{z}^{[i-1]} + \bm{b}^{[i]}\right) ,\,i=1,\ldots,L \\
    f_\theta(\bm{x}) &= \bm{z}^{[L]}
,\end{align*}
in which $\theta=\left( \theta^1,\ldots,\theta^L \right) $ are the vectors of parameters and $g^{[i]}$ are the activation functions. It is common to see $\theta$ and each of the $\theta^i$ as a vector composed of the individual elements of the respective $A^{[i]}$ and $\bm{b}^{[i]}$. Using this notation, we can say that, given a \textit{target} function $f^*$ and input data $X$, a deep feedforward network $f_\theta$ must learn a set of parameters $\theta$ such that $f_\theta(x) \approx f^*(x),\,\forall x \in X$.

Let us recall the Exclusive-OR example from the previous section. One can construct a two-layer deep feedforward network such that the functions $f_\theta^{[1]}:\R^2\to\R^2$ and $f_\theta^{[2]}:\R^2\to\R$ are
\begin{align*}
    f_\theta^{[1]}(\bm{z}) &= \sigma\left(
    \begin{bmatrix}
    2K & 2K \\
    -2K & -2K
    \end{bmatrix}\bm{z} + \begin{bmatrix}
    -K \\
    3K
    \end{bmatrix}\right) \\
    f_\theta^{[2]}(\bm{z}) &= \sigma\left(
    \begin{bmatrix}
    2K & 2K
    \end{bmatrix}\bm{z} - 3K\right)
,\end{align*}
where the chosen activation function $\sigma(.)$ is the sigmoid function\footnote{Applied element-wise where necessary.} and the parameters are defined from $K \gg 1$ such that $\sigma(K) \approx 1$ and $\sigma(-K) \approx 0$.
Then, it is easy to see that the first output of $f^{[1]}$ approximates the OR function applied to the input, while the second output approximates the NAND function and $f^{[2]}$ approximates an AND function.

For a network defined as previously, we say that the inner states $\bm{z}^{[i]}$ which are neither the input nor the output of the network (i.e., $i\not\in \{0,L\}$) are called the \textit{hidden layers} of the network.
Note that even though the input and output dimensions are defined by the target function, the dimensions of the hidden layers are a design choice, as well as the number of hidden layers and the activation functions.

It has already been proven that a network with a single hidden layer can approximate any continuous function on a closed and bounded subset of $\R^n$, for a broad range of activation functions, given that the hidden layer has enough dimensions \cite{hornik_multilayer_1989,leshno_multilayer_1993}.
Yet, even though this universal approximation theorem guarantees that such a network exists, it provides no way to find it.
In practice, a network with a single hidden layer might need to be unfeasibly large to achieve the desired approximation, while deeper models can be far more efficient \cite{goodfellow_deep_2016}.

\section{Learning}

The Exclusive-OR example illustrates well how a model built of multiple simple functions can approximate very well a target behavior.
Yet, if we consider complex tasks (such as predicting age from humans' photographs), it is easy to see that many components and hundreds, maybe millions of parameters may be necessary\footnotemark.
Thus, it is not always reasonable (or even feasible) to design these components manually.
That is the reason \textit{learning algorithms} are essential to make these models useful for realistic tasks.

\footnotetext{A simple example would be to consider a deep feedforward network designed to have images as inputs and output a single value. Even if small, 32-by-32 pixels, grayscale images are expected, the domain will have dimension 1024. Thus, if a single hidden layer of dimension 256 (a quarter of the input size) is desired before the output layer, the network will have over 200 thousand parameters.}

One can understand what a learning algorithm is from the definition from \textcite{mitchell_machine_1997}:
``A computer program is said to learn from experience $E$ with respect to some class of tasks $T$ and performance measure $P$, if its performance at tasks in $T$, as measured by $P$, improves with experience $E$.''
To narrow down this broad definition to the scope of the presented work, we can consider the tasks in $T$ as target functions which we want our model to approximate.
For this class of tasks, it is natural that the performance measure $P$ be some sort of distance measure between the model and the target function.

It is often impractical to evaluate both the model and the target function in the entirety of the domain desired.
Both because such evaluation can be too expensive to compute, and because the behavior of the target function might not be known beforehand.
Thus, the experience $E$ usually comes from data in the form of a (finite) set of samples from the domain paired with the outputs of the target function of interest.
% The performance is also usually measured on the data, discarding the need to evaluate the target function and the model on the entirety of the domain.

Roughly speaking, the deep learning algorithms of interest for this work are those that, given data on the target function, provide a deep learning model that maximizes a performance measure.
This is usually called \emph{training} a model.

\subsection{Gradient Descent}

The majority of deep learning algorithms of our interest require some sort of optimization.
It would be natural to frame the definition of a learning algorithm as an optimization problem that maximizes the performance measure.
Yet, not always the performance measure of interest is easy to compute or provides useful features for the optimization (e.g, differentiability).
Therefore, it is usual to optimize indirectly, minimizing a \textit{loss function} while aiming to improve the performance measure.

Given a model $\bm{f}_\theta:\R^n\to\R^m$ and a target function $\bm{f}^*:U\subset\R^n\to\R^m$, an ideal performance measure could be $\int_U \|\bm{f}_\theta(\bm{x}) - \bm{f}^*(\bm{x})\|d\bm{x}$.
Yet, this integral could be costly to compute and might not even be easily defined.
Instead, one could evaluate the model on a finite set of points from the domain $X=\{(\bm{x},\bm{y})\in U\times\R^m : \bm{y}=\bm{f}^*(\bm{x})\}$ (the data that forms the experience $E$), defining a loss function $l:\R^m\times\R^m\to\R$ (over the model's output and the target) that can be evaluated for each sample, e.g., the $\ell^2$-norm.
This way, the outcome of a deep learning algorithm would be a model $\bm{f}_{\theta^*}$, such that \[
\theta^* = \arg\min_\theta \sum_{(\bm{x},\bm{y})\in X} l(\bm{f}_\theta(\bm{x}), \bm{y})
.\] Usually, though, given the data $X$ and the model $\bm{f}_\theta$, one defines a \emph{cost function} $J: \Theta\to\R$, which can be an aggregation of the per-sample loss function like $J\left( \theta \right) = \sum_{(\bm{x},\bm{y})\in X} l(\bm{f}_\theta(\bm{x}), \bm{y})$.
Therefore, to train a model one can solve the optimization problem
\begin{align*}
    \min_\theta \quad & J\left( \theta \right)  \\
    \text{s.t.} \quad & \theta \in \Theta
,\end{align*}
where $\Theta$ is the set of feasible parameters for the model.

The most common way to solve this optimization for deep learning models is to use the \textit{gradient descent} algorithm.
This method was first proposed by Cauchy in the XIX century \cite{lemarechal_cauchy_2012} based on the definition of the gradient of a differentiable function.
It is known that, given a differentiable function $f:A\to\R$ and $\bm{a} \in A$, if $\| \nabla f(\bm{a}) \| \neq 0,\,\exists \gamma > 0$ such that $f(\bm{a} - \gamma \nabla f(\bm{a})) < f(\bm{a})$, that is, if one takes a small enough step in the opposite direction of the gradient, the function is certain to decrease.
The scalar $\gamma$ is called the \textit{learning rate}, and defines the size of the step that is taken \cite{goodfellow_deep_2016}.

Therefore, following the notation of the previous example, if we assume that both the model and the cost function are differentiable, given parameters $\theta_k$ such that $\| \nabla J(\theta_k) \| \neq 0$, if we set
\begin{equation}\label{eq:gradient-descent-step}
\theta_{k+1} \gets \theta_k - \gamma \nabla J(\theta_k) 
,\end{equation}
then, for a sufficiently small $\gamma$, we know that \[
J\left( \theta_{k+1} \right) < J\left( \theta_k \right) 
,\] i.e., the cost will decrease.
Notice how intuitively it seems that if one takes enough steps like \eqref{eq:gradient-descent-step} with small enough $\gamma$, the parameters will converge to a point that is a local minimum of the cost function.

% Maybe introduce the gradient step with regards to the cost function = sum of per-sample loss function?

Yet, gradient descent provides poor convergence conditions \cite{wolfe_convergence_1969}, being considered unreliable and slow for many practical problems.
Still, gradient descent is known to work very well for deep learning.
In this area, it has shown to achieve low values of the cost function fast enough to be useful, even if it does not find a local minimum \cite{goodfellow_deep_2016}

\subsection{Gradient Descent Variations}

Over the year, many improvements on the gradient descent algorithms made it even more reliable and efficient for learning deep networks.
In particular, the use of \textit{momentum} on the update of the parameters.
Computing a velocity vector over the parameter update steps, and taking this into account when performing the update, has shown to accelerate significantly the learning process \cite{sutskever_importance_2013}.
The classical way of doing this is to replace the update in \eqref{eq:gradient-descent-step} by
\begin{align*}
    v_{k+1} &\gets \mu v_k - \gamma \nabla J(\theta_k) \\
    \theta_{k+1} &\gets \theta_k + v_{k+1}
,\end{align*}
where $v_k$ is the velocity at the $k$ step, and $\mu$ is the momentum coefficient.

A major challenge of the gradient descent as a learning algorithm is that it introduces a parameter that is not learned in the process (which are commonly called \emph{hyperparameters}) and that is crucial to finding a good model: the learning rate.
Using momentum reduces a little the sensitivity of the results to the choice of the learning rate, but does so while introducing a new hyperparameter (the momentum coefficient $\mu$).
Furthermore, it is known that the cost function can be sensitive to some of the parameters while being numb to others \cite{goodfellow_deep_2016}.

It was natural, in the face of these challenges, to think of strategies that use different learning rates for each parameter and automatically change these learning rates throughout the learning process.
The most simple way of doing this is following the idea that, if the gradient with respect to a given parameter does not change sign (i.e., remains positive (or negative) over the iterations), its associated learning rate should increase; otherwise, it should decrease.
This way, if the cost function is sensitive to a given parameter, its associated learning rate will decrease over time, making the training less unstable.

Adam\cite{kingma_adam_2015} combined two gradient descent variations proposed by \textcite{duchi_adaptive_2011} and \textcite{tieleman_lecture_2012}, encompassing both adaptive learning rates and momentum, and taking them one step further.
After showing excellent empirical results over many application areas, Adam became one of the most commonly used algorithms and is seen as one of the best choices overall \cite{ruder_overview_2017}.

\section{Regularization}

One of the biggest challenges in deep learning is to get models that perform well also on samples that are no in the data provided during training, i.e., models that \emph{generalize} well.
The strategies designed to improve performance outside of the data seen during training at the cost of decreased performance in the training data are known as \emph{regularization} \cite{goodfellow_deep_2016}.

One of the most common regularization practices is to penalize the magnitude of the parameters by adding a term to the cost function as $J\left( \theta \right) = \sum_{(\bm{x},\bm{y})\in X} l(\bm{f}_\theta(\bm{x}), \bm{y}) + \alpha\Omega\left( \theta \right) $, where $\alpha$ is a hyperparameter that weighs the contribution of the new term and $\Omega:\Theta\to\R$ can be, e.g., the $\ell^1$-norm.
It can be seen as an incentive for the model to be built on simple features, in an effort to approximate the training data with the simplest model possible \cite{goodfellow_deep_2016}.
% A parallel can be traced with the task of interpolating data acquired sparsely: high frequency (complex) terms can be used to interpolate the data, but these might not describe the data generating function.
One may ponder why this may be useful, when limiting the complexity of the model through hard constraints is rather easy (e.g., limiting the number of parameters, reducing $\Theta$, etc.).
Unfortunately, properly defining these hard constraints is not trivial, as finding the ``simplest'' model that is still able to approximate the target function can be hard.
Furthermore, in practice, more complex models properly regularized almost always perform better than simpler models \cite{goodfellow_deep_2016}.

Another way to improve generalization is to use regularization to drive the learning towards models that present some desired characteristics.
One of the most common of such characteristics is noise robustness.
If one wants to train a model that is robust to noise in its input, one way to achieve this is to reduce the output's sensitivity.
The output can be said less sensitive to variations of an input dimension if its derivative with respect to this input dimension has a small magnitude.
This can be enforced through a regularization term on the gradients of the model's outputs \cite{drucker_improving_1992}.
E.g., given a model $f_\theta:\R^n\to\R$, one can use a regularization term of the form \[
    \Omega\left( \theta \right) = \sum_{\left( \bm{x},y \right) \in X} \| \nabla f_\theta\left( \bm{x} \right) \|
.\] Note, however, that to use gradient descent with this type of regularization, one must be able to compute second-order derivatives of $f_\theta$.

\section{Back-Propagation}

Even from the most simplistic description of the gradient descent algorithm, as in equation \eqref{eq:gradient-descent-step}, it is clear that the challenge lies in computing the gradient of the cost function.
Taking deep feedforward networks as an example, the analytical formula for this gradient can be derived without much effort.
Yet, evaluating this formula can be quite expensive, given that, as already seen, deep learning models can have millions of parameters, thus making even the computation of a linear application non-trivial.
The \emph{back-propagation} algorithm \cite{rumelhart_learning_1986} provides a clever way of computing the gradient with respect to each parameter without great computational costs.

Back-propagation works based on the \emph{chain-rule} for the derivatives.
We first note that, from \eqref{eq:gradient-descent-step} and the definition of the cost function, $\nabla J$ can be reduced to evaluating the gradient of the loss function at several points.
Therefore, if we want to compute the derivative of the cost function with respect to each of the parameters $\theta^i,\,i=1,\ldots,L$, we need to look at the derivative of the loss function with respect to these parameters.
Now, by the chain-rule, let us take the case for $\theta^L$ and see that
 \begin{align*}
     \frac{\partial l}{\partial\theta^L} &= \nabla_{f_\theta} l \frac{d f_\theta}{d \theta^L} \\
     &= \nabla_{f_\theta} l \frac{d f^L_{\theta^L}}{d \theta^L}
,\end{align*}
where $\nabla_{f_\theta} l$ is the gradient of the loss function with respect to the model's output and $\frac{d f^L_{\theta^L}}{d \theta^L}$ is the Jacobian of the last layer of the model. Now, if we consider the case for $\theta^{L-1}$ and $\theta^{L-2}$
 \begin{align*}
     \frac{\partial l}{\partial\theta^{L-1}} &= \nabla_{f_\theta} l \frac{d f^L_{\theta^L}}{d f^{L-1}_{\theta^{L-1}}} \frac{d f^{L-1}_{\theta^{L-1}}}{d \theta^{L-1}} \\
     \frac{\partial l}{\partial\theta^{L-2}} &= \nabla_{f_\theta} l \frac{d f^L_{\theta^L}}{d f^{L-1}_{\theta^{L-1}}} \frac{d f^{L-1}_{\theta^{L-1}}}{d f^{L-2}_{\theta^{L-2}}} \frac{d f^{L-2}_{\theta^{L-2}}}{d \theta^{L-2}}
,\end{align*}
it is easy to see repeating terms, that is, the result of $\nabla_{f_\theta} l \frac{d f^L_{\theta^L}}{d f^{L-1}_{\theta^{L-1}}}$, necessary to compute $\frac{\partial l}{\partial\theta^{L-1}}$ can be reused to compute $\frac{\partial l}{\partial\theta^{L-2}}$ and all of $\theta^{L-3},\ldots,\theta^1$.

Back-propagation exploits this by computing the gradients with respect to each parameter "from left to right", that is, starting by computing the gradient of the outermost component of the composition and storing the intermediate result.
Applied to the equations above, the first operation would be to compute $\bm{u}\gets\nabla_{f_\theta} l$, which would be used in the vector-Jacobian product $\bm{u} \frac{d f_\theta}{d \theta^L}$ to compute the gradient with respect to $\theta^{L}$.
Then, the intermediate result can be updated through the vector-Jacobian product $\bm{u}\gets \bm{u} \frac{d f^L_{\theta^L}}{d f^{L-1}_{\theta^{L-1}}}$, which is useful to compute the gradient with respect to $\theta^{L-1}$.

It is easy to see that this procedure can be repeated, propagating the gradient through each layer of the network until every parameter is covered, effectively reducing the evaluation of the gradient to computing several vector-Jacobian product.
Furthermore, back-propagation can be also very memory efficient, as the chain-rule can be applied to each $f^i_{\theta^i}$ and, thus, the stored values are limited by the layer size\footnotemark.
\footnotetext{The presented approach considers the $\theta^i$ vectors as the parameters of interest, when in a practical application the gradients would be computed with respect to each of the $A^{[i]}$ and $\bm{b}^{[i]}$ parameters. Therefore, the Jacobian matrices computed would be limited by the sizes of each of these parameters. Finally, it is easy to see that the intermediate value stored $\bm{u}$ will have the same dimension as the next layer used in the vector-Jacobian product to update it.}

% % The great results that deep learning techniques have achieved come not only from the great theoretical guarantees, but also from the practical considerations that make these models an efficient alternative.
% 
% Even from the most simplistic description of the gradient descent algorithm (as in equation \eqref{eq:gradient-descent-step}) it is clear that its computational cost comes majorly from computing derivatives.
% For this reason, the efficiency of performing this operation is a major driver of deep learning's success.
% Even though it is not the only alternative, \emph{automatic differentiation} is the most successful technique for deep learning, being used by the two most used packages[REFTO Pytorch \& TensorFlow].
% % This pairing comes from the flexibility of automatic differentiation to computing derivatives of compositions of many functions, which is in deep learning's core.


% ---

% ---
% 4 - PINN
% ---
% ----------------------------------------------------------
\chapter{Physics-Informed Learning}\label{ch:pinn}
% ----------------------------------------------------------

Solving an \gls{IVP} for \gls{ODE}s using deep learning is not as straight-forward as it may seem.
Indeed, it falls under the function approximation paradigm, in which we want to train a parametrized model to approximate a target function (which is the solution to the \gls{IVP}).
In this scenario, it is usual that either the target function is unknown or it is too complex to be useful, and, thus, only input-output samples are available to train the deep learning model.
This is the case for most of the well-known successful applications of deep learning, such as those involving computer vision and the classification of images. % TODO find sources

However, that is not the case here.
When a solution for an \gls{IVP} is desired, the target function is not known, but neither are the input-output pairs.
Actually, generating the training data is essentially solving the \gls{IVP}.
Yet, the \emph{dynamics} of the solution is known through the \gls{ODE}.
This chapter focuses on an approach to harvest this knowledge efficiently and use it to train a deep feedforward network that approximates the solution of an \gls{IVP}.
More specifically, the work of \textcite{Raissi2019} is presented in a more limited formulation, targeting \gls{ODE}s\footnotemark.
\footnotetext{As the original work is for partial differential equations, having \gls{ODE}s as a particular case.}

\section{Problem Statement}

Let us first recall the \gls{IVP}.
Given an \gls{ODE} such as the one defined in \eqref{eq:ode} and boundary conditions, we want to find a solution satisfies both.
More precisely, given $\mathcal{N}:\R\times \R^m\to \R^{m}$ and initial conditions $t_0\in I\subset \R,\,\bm{y}_0\subset \R^{m}$, we want to find $\bm{\phi}:I\to \R^{m}$ such that
\begin{align*}
    \frac{d \bm{\phi}(t)}{d t} &= \mathcal{N}\left( t, \bm{\phi}(t) \right),\,t\in I \\
    \bm{\phi}(t_0) &= \bm{y}_0
\end{align*}
is true.

Solving this using deep learning means to train a model $\bm{f}_{\theta}:I\to \R^{m}$, parametrized by $\theta\in \Omega$, that satisfies the same conditions, i.e.,
\begin{align}
    \frac{d \bm{f}_\theta(t)}{d t} &= \mathcal{N}\left( t, \bm{f}_\theta(t) \right),\,t\in I \label{eq:dl-ode} \\
    \bm{f}_\theta(t_0) &= \bm{y}_0 \label{eq:dl-ivp}
.\end{align}

The naïve approach would be to construct a set $X=\left\{ (t,\bm{y})\in I\times \R^{m}: \bm{y}=\bm{\phi}(t) \right\} $ to be used as the experience for the learning algorithm.
Then, with a properly constructed cost function and given that the model is complex enough, the model $\bm{f}_\theta$ would approximate the target $\bm{\phi}$ and, by consequence, satisfy equations \eqref{eq:dl-ode} and \eqref{eq:dl-ivp}.
However, note how this approach assumes that $\bm{\phi}$ is known, as it is required to construct a set $X$ with more than just $\left( t_0,\bm{y}_0 \right) $.
Therefore, in many \gls{IVP} setups, training a deep learning model this way would either be impossible or redundant.

\section{Physics Regularization}

The naïve approach described above is quite inefficient in that it does not use the information provided by the known $\mathcal{N}$ function.
This is precisely the turning point for making deep learning a viable option in solving \gls{IVP}s.
The approach of \textcite{Raissi2019} proposes to train the model using a regularization based on $\mathcal{N}$.
This means to train $\bm{f}_\theta$ to approximate $\bm{\phi}$ at the initial condition (since this is known by the definition), satisfying equation \eqref{eq:dl-ivp}, and so that it's Jacobian approximates $\mathcal{N}$, satisfying equation \eqref{eq:dl-ode}.

For this, let us define the singleton $X_b=\left\{ \left( t_0,\bm{y}_0 \right)  \right\} $ and the set $X_{\mathcal{N}}=\left\{ t\in I \right\} $. Then, we can construct \[
    J_b\left( \theta \right) = \sum_{\left( t,\bm{y} \right) \in X_b } \|\bm{f}_\theta(t) - \bm{y}\|_2 = \|\bm{f}_\theta\left( t_0 \right) -\bm{y}_0\|_2
,\] which looks like a usual cost function, and \[
J_{\mathcal{N}}\left( \theta \right) = \sum_{t \in X_{\mathcal{N}}} \left\| \frac{d \bm{f}_\theta\left( t \right) }{dt} - \mathcal{N}\left( t,\bm{f}_\theta\left( t \right)  \right)  \right\|_2
,\] which resembles a gradient regularization as discussed in sec. \ref{sec:regularization}, the difference here is that instead of penalizing high derivatives, we want them to approach a desired value.
Then, the cost function used to train the model is defined as \[
J\left( \theta \right) = J_b\left( \theta \right) + \lambda J_{\mathcal{N}}\left( \theta \right) 
,\] where $\lambda \in \R^+$ is a scalar value to weight in the components of the loss function.
Originally, \textcite{Raissi2019} defines this value as $\lambda = |X_{\mathcal{N}}|^{-1}$, that is, the inverse of the number of elements in the $X_{\mathcal{N}}$ set\footnotemark.
\footnotetext{This is for the particular case of $|X_b|=1$, which makes the "contribution" of the components proportional to the respective set size.}
A neural network trained to learn a differential equation following a cost function as the one defined above is called a \gls{PINN}.

The intuition of this approach is that $J_b$ will guide the optimization so that equation \eqref{eq:dl-ivp} is satisfied, while  $J_{\mathcal{N}}$ will ponder it towards satisfying \eqref{eq:dl-ode}.
Unfortunately, no theoretical guarantees have been published to prove this intuition.
Nevertheless, the authors have provided plenty of empirical evidence together with a robustness analysis, indicating that with enough samples in $X_{\mathcal{N}}$ and a sufficiently complex model $\bm{f}_\theta$, a small error (e.g., $\|\bm{f}_\theta\left( t \right)-\bm{\phi}\left( t \right) \| $) can be achieved \cite{Raissi2019}.
This result has also been achieved by others in different applications, validating the claims \cite{noakoasteen_physics-informed_2020,zhang_physics-informed_2020,Arnold2021,Yucesan2022}.

Finally, note how this approach does not require that the target function is known.
Actually, $X_{\mathcal{N}}$ can be constructed randomly by extracting samples of $I$,
therefore, making deep learning an efficient approach to solving \gls{IVP}s.

\subsection*{Example}

Let us apply the approach presented above and train a \gls{PINN} to solve an \gls{IVP} for Newton's second law of motion.
Recall that it can be modeled as a first-order \gls{ODE} of the form
\[
    \frac{d \bm{y}(t)}{dt} = \begin{bmatrix} \frac{d y_1(t)}{dt} \\ \frac{y_2(t)}{dt} \end{bmatrix} = \begin{bmatrix} y_2(t) \\ \frac{C}{M} \end{bmatrix} 
,\]
assuming that the force applied to the object is a constant $C$.
For simplicity, let us assume that $C=M=1$
Finally, let us define the \gls{IVP} through $t_0=0$ and $\bm{y}_0=\left( 0,0 \right) $, that is, at the initial time, the object stands still at the reference position.
Also, the interval in which we are interested is $I=\left[ 0,1 \right] $.

Now, for the deep learning model, let us use a deep feedforward network with 2 hidden layers of size 10, i.e., our model is a function $\bm{f}_\theta:\R\to \R^2$ such that
\[
    \bm{f}_\theta(t) = \bm{f}_\theta^{[3]} \circ \bm{f}_\theta^{[2]} \circ \bm{f}_\theta^{[1]} \left( t \right) 
,\] in which $\bm{f}_\theta^{[1]}:\R\to \R^{10}$, $\bm{f}_\theta^{[2]}:\R^{10}\to \R^{10}$, $\bm{f}_\theta^{[3]}:\R^{10}\to \R^2$, and $\bm{f}_\theta^{[i]}\left( \bm{z} \right) =\tanh\left( A^{[i]}\bm{z} + \bm{b}^{[i]} \right) $.
In the results shown in figure \ref{fig:images-pinn_newton-pdf}, this model was implemented using PyTorch[REFTO Paszke, 2019] and trained using Adam with $\gamma = 0.1$ and $\lambda=0.1$.
Running the algorithm for 1000 epochs took less than 2 seconds on a high-end, 16-core processor (no GPU was used).
Notice how the \gls{PINN} evolves over the epochs, achieving the same results of the numerical solver even though only the dynamics and the initial condition were experienced during training.

\begin{figure}[h]
    \centering
    \includegraphics[width=0.8\textwidth]{images/pinn_newton.pdf}
    \caption{Performance of a PINN in comparison to \gls{RK4} in solving an \gls{IVP} of Newton's second law for $I=\left[ 0,1 \right] $. In the graphic, $y_1$ and $y_2$ are the results of the numerical solver, while $\hat{y}_1$ and $\hat{y}_2$ are the results of the PINN trained after different number of epochs.}
    \label{fig:images-pinn_newton-pdf}
\end{figure}


% ---

% ---
% 5 - DEQ
% ---
% ----------------------------------------------------------
\chapter{Deep Equilibrium Models}\label{ch:deq}
% ----------------------------------------------------------

This chapter is dedicated to lay out the foundations of the model architecture that is the central investigation point of this work.
\gls{DEQ}s have been proposed by \textcite{Bai2019} and \textcite{Ghaoui2019}, the latter naming them \emph{implicit models}.
In this chapter, we follow the notation of the former.

Furthermore, one of the greatest challenges in working with \gls{DEQ}s is that, by their implicit nature, they do not fit perfectly well with current deep learning tools.
Therefore, to better understand the nuances and challenges that this family of models present during the experiments, a good share of attention is dedicated to the specificities of performing back propagation with \gls{DEQ}s.

\section{Introduction and Definition}

In Chapter \ref{ch:deep-learning}, the intuition behind a deep learning model was introduced, that is, to model complex features through the composition of simple-yet-non-linear parametrized functions.
Besides the network defined in \ref{sec:neural-nets} (which is the base for \gls{PINN}s, as shown in chapter \ref{ch:pinn}), many other deep learning model architectures have been proposed over the years.
    Some of the architectures with the most surprising results involve composing the models with the same function applied multiple times, i.e., following the notation of chapter \ref{ch:deep-learning}, instead of defining the model as $f_{\gls{param}}=f_{\gls{param}}^{[L]}\circ \cdots \circ f_{\gls{param}}^{[1]}$, these architectures suggest a model similar to $f_{\gls{param}}=f_{\gls{param}}^{[1]}\circ \cdots \circ f_{\gls{param}}^{[1]}$.
Inspired by this, \textcite{Bai2019} takes a step further, defining the model with a (possibly) infinite stack of the same function, which was named \gls{DEQ}.

Let us recall the definition of a deep learning model as proposed in equation \eqref{eq:dl-model}, but imagine it has an infinite number of layers (infinite depth).
Of course, if each $f^{[i]}$ is a different function (with different parameters), then this would be impossible to fit in memory.
Therefore, let us assume that $f_{\gls{param}}^{[i]}=f_{\gls{param}}^{[EQ]},\forall i$, i.e.,
\begin{equation*}
\begin{split}
    z^{[0]} &= x \\
    z^{[i]} &= f_{\gls{param}}^{[EQ]}(z^{[i-1]}), \forall i\ge 1
,\end{split}
\end{equation*}
in which the output would be $z^{\star} = z^{[\infty]}$.
If this iterative process converges, that is, if there is a number $N$ such that $\forall i\ge N,\,z^{[i]}\approx z^{[i+1]}$, then the output $z^{\star}\approx z^{[N]}$ is well-defined, and it is true that  \[
    z^{\star} = f_{\gls{param}}\left( z^{\star} \right) 
.\] 
One can say that $z^{\star}$ is an \emph{equilibrium point} of $f_{\gls{param}}^{[EQ]}$.
Therefore, the output of a well-behaved (i.e., one that respects the restrictions above presented) infinite-depth deep learning model can be computed by finding its equilibrium point.

The model proposed by \textcite{Bai2019} has a slight change in how it handles the input, feeding the input vector at each layer of the model. More precisely, we can say that a \gls{DEQ} of the form
\begin{align*}
    \bm{f}_{\gls{param}}: \R^{n} &\longrightarrow \R^{m} \\
    \bm{x} &\longmapsto \bm{f}_{\gls{param}}(\bm{x}) = \bm{z}^{\star}
\end{align*}
defines its output as the equilibrium point of a function $\bm{f}_{\gls{param}}^{[EQ]}:\R^{n+m}\to \R^{m}$
\begin{equation}\label{eq:z-star}
    \bm{z}^{\star} = \bm{f}_{\gls{param}}^{[EQ]}\left( \bm{x},\bm{z}^{\star} \right) 
.\end{equation}

\section{Forward}

The most simple way to perform the forward pass of a \gls{DEQ}, i.e., to compute the output of the model given an input, is to iterate the application of the equilibrium function $\bm{f}_{\gls{param}}^{[EQ]}$ until the current value is close enough to the previous.
More specifically, given an input $\bm{x}$ and an initial guess $\bm{z}^{[0]}$, the procedure is to update the equilibrium guess $\bm{z}^{[i]}$ by \[
    \bm{z}^{[i]} = \bm{f}_{\gls{param}}^{[EQ]}(\bm{x}, \bm{z}^{[i-1]})
\] until $\|\bm{z}^{[i]}-\bm{z}^{[i+1]}\|$ is small enough.
This approach is the \emph{simple iteration} method\cite{suli_introduction_2003}.
Even though this approach is very intuitive given our derivation of a \gls{DEQ} from an infinite-depth model, it is quite limited.
First because it can be quite slow, i.e., it can take many iterations until convergence is achieved, being very sensitive to the starting point.
But mostly because this approach only finds equilibrium points if the function of interest is a contraction between the starting point and the equilibrium point\cite{suli_introduction_2003}.

This limitation can be easily visualized by trying to use the simple iteration method to find the equilibrium of $f(z) = 2z-1$.
The function clearly has an equilibrium at $z=1$.
Yet, at any starting point \emph{except} the equilibrium, the simple iteration method will diverge.

Luckily, we know from equation \eqref{eq:z-star} that the equilibrium point, for a given input, is also the root of a function $\bm{g}_{\bm{x}}(\bm{z}) = \bm{f}_{\gls{param}}^{[EQ]}(\bm{x},\bm{z}) - \bm{z}$.
This means that using any root-finding algorithm on $\bm{g}_{\bm{x}}$ yields $\bm{z}^{\star}$, the desired output.
Perhaps the most classical root-finding algorithm is \emph{Newton's method}, which proposes to iterate over the solution space given \[
    \bm{z}^{[i+1]} = \bm{z}^{[i]} - \left( \frac{d \bm{f}_{\gls{param}}^{[EQ]}(\bm{x},\bm{z}^{[i]})}{d\bm{z}} \right)^{-1} \bm{f}_{\gls{param}}^{[EQ]}(\bm{x},\bm{z}^{[i]})
,\] 
in which $\frac{d \bm{f}_{\gls{param}}^{[EQ]}(\bm{x},\bm{z}^{[i]})}{d\bm{z}}$ represents the Jacobian of  $\bm{f}_{\gls{param}}^{[EQ]}$ with respect to $\bm{z}$\footnotemark.
\footnotetext{To avoid the computational burden of inverting the Jacobian matrix, it is usual that the iteration focuses instead in solving $ \frac{d \bm{f}_{\gls{param}}^{[EQ]}(\bm{x},\bm{z}^{[i]})}{d\bm{z}} \left(\bm{z}^{[i+1]} - \bm{z}^{[i]}\right) = -\bm{f}_{\gls{param}}^{[EQ]}(\bm{x},\bm{z}^{[i]})$ instead.}
Newton's method not only has guaranteed convergence for a broader class of functions in comparison to simple iteration, but also converges much faster\cite{suli_introduction_2003}.

\subsection{Practical Considerations}

Most modern algorithms that help us find the desired equilibrium point are either modifications of the simple iteration algorithm (e.g., Anderson Acceleration\cite{walker_anderson_2011}) or modifications of Newton's method (e.g., Broyden's method\cite{broyden_class_1965}).
Nevertheless, all these methods require an initial guess $\bm{z}^{[0]}$ and a tolerance $\epsilon>0$. 
The initial guess, or starting point, is clearly necessary, as it is natural for iterative procedures, and is usual to find it as $\bm{z}^{[0]}\gets \bm{0}$ by default.
The tolerance is necessary to define a stopping condition for the algorithm, when the approximation for the equilibrium point is "good enough", i.e., if $\|\bm{z}^{[i]}-\bm{z}^{[i+1]}\|<\epsilon$ then it is considered that the equilibrium has been reached.
Furthermore, it is also usual to define a limit for the number of iterations, avoiding that the algorithms run for an indefinite amount of time.

\subsection{Jacobian Regularization}

Two common empirical findings of \gls{DEQ} applications are that they are 1) unstable to architectural choices \cite{bai_stabilizing_2021} and 2) increasingly slower over training iterations \cite{Bai2019,winston_monotone_2020}.
This is a direct implication of the equilibrium-finding nature of the forward pass, which relies heavily on the behavior of $\bm{f}_{\gls{param}}^{[EQ]}$.
Intuitively, the complexity of $\bm{f}_{\gls{param}}^{[EQ]}$, which depends heavily on the architecture and is expected to increase during training, makes it harder for the root-finding algorithm to converge.

In a very recent work, \textcite{bai_stabilizing_2021} discussed how the Jacobian of $\bm{f}_{\gls{param}}^{[EQ]}$ with respect to $\bm{z}$ is related to both problems.
The authors propose, then, to penalize large values in this Jacobian during training and show how this increases robustness and convergence speed of \gls{DEQ}s, reducing training and inference times.
More specifically, they propose to compute the Frobenius norm\footnotemark of the Jacobian of $\bm{f}_{\gls{param}}^{[EQ]}$ and add it as a regularization term to the cost function (see sec. \ref{sec:regularization}).
\footnotetext{The Frobenius norm of a matrix $A$ can be written $\|A\|_F=\sqrt{\sum_{i,j=1}^{n} |a_{i,j}|^2} $.}

\section{Backward}

It was shown in sec. \ref{sec:backprop} that, in order to use a gradient descent algorithm to train a deep learning model, one must compute the derivatives of the cost function with respect to the model's parameters.
In the case of \gls{DEQ}s, this computation is not straight-forward.
Given $\bm{f}_{\gls{param}}:\R^{n}\to \R^{m}$ a \gls{DEQ} as defined above, computing the derivative of a cost function $J:\Omega\to \R$ with respect to the parameters can be seen as \[
    \nabla_{\gls{param}} J\left( \gls{param} \right) = \nabla_{\bm{f}_{\gls{param}}} J \frac{d \bm{f}_{\gls{param}}(\bm{x})}{d\gls{param}}
.\] 
At the same time, we know that the output of a model is given by an equilibrium point of $\bm{f}_{\gls{param}}^{[EQ]}$, which is computed using a root finding method, i.e., \[
    \bm{f}_{\gls{param}}(\bm{x}) = RootFind(\bm{g}_{\bm{x}}, \bm{z}^{[0]})
,\] where $\bm{g}_{\bm{x}}$ is a function so that $\bm{g}_{\bm{x}}(\bm{z}) = \bm{f}_{\gls{param}}^{[EQ]}(\bm{x},\bm{z}) - \bm{z}$.
Therefore, computing the derivatives of $\bm{f}_{\gls{param}}$ directly, requires the computation of the derivatives of the root-finding algorithm.
Yet, not only root-finding algorithms need not be differentiable, but even those that are may impose an enormous computational burden to compute the actual derivatives.
To illustrate the point, if we restrain ourselves to using the simple iteration method as the root-finding algorithm, we can apply the chain rule and decompose the derivative in computing the derivative of $\bm{f}_{\gls{param}}^{[EQ]}$ for as many times as there were iterations until convergence, which can make practical applications impossible.

Luckily, we can exploit the fact that the output of the model ($\bm{f}_{\gls{param}}\left( \bm{x} \right) = \bm{z}^{\star}$) is an equilibrium point of $\bm{f}_{\gls{param}}^{[EQ]}$.
This implies that, in a neighborhood of the input vector $\bm{x}$, $\bm{f}_{\gls{param}}$ is a \emph{parametrization} of \bm{z} with respect to $\bm{x}$, i.e., \[
    \bm{f}_{\gls{param}}^{[EQ]}\left( \bm{x},\bm{f}_{\gls{param}}(\bm{x}) \right) -\bm{f}_{\gls{param}}\left( \bm{x} \right) = \bm{0}
\] is true.


- recall backprop
- show how it plays out for DEQs
- root-finding need not be differentiable
- model is a parametrization of fEQ
- implicit function theorem

\subsection{Implementation}

- in practice, computing the inverse is expensive
- luckily, we just need the vjp
- turns vjp can be found through root finding!


% ---

% ---
% 6 - Experiments and Results
% ---
% ----------------------------------------------------------
\chapter{Experiments and Results}
% ----------------------------------------------------------

\section{PIDEQ?}

- definition of PIDEQ

\section{Methodology}

\subsection{Data}

\subsection{Training}

\subsection{Evaluation Metrics}

\section{Results}

% ---

% ---
% 7 - Conclusão
% ---
%\phantompart
% ----------------------------------------------------------
\chapter{Conclusion}\label{ch:conclusion}
% ----------------------------------------------------------

An in-depth study was carried out with two novel approaches in the big area of deep learning: \gls{PINN} and {DEQ}.
Both provide a deeper connection between deep learning and other areas of knowledge, such as differential equations, optimization, and numerical analysis.
\gls{PINN}s provide an efficient approach to train deep learning models on problems involving physical phenomena.
\gls{DEQ}s are a promising new architecture that can provide a larger representational power with a small number of parameters.
This work proposed an application that combines both: using \gls{PIDEQ}s to solve \gls{IVP}s of \gls{ODE}s.
For this, we successfully studied, implemented and tested several \gls{PIDEQ} models.

To the best of our knowledge, this is the first study on physics regularization (or any gradient regularization) for \gls{DEQ}s.
In fact, we have not found any published result reporting higher-order derivatives of these models, as \textcite{Bai2019} only proposed an analytical solution for the first derivative.
The requirement for higher-order derivatives imposes limitations to the model, namely the use of a differentiable solver to compute the first derivative.

To validate the implementation of the model, we trained \gls{PIDEQ}s to solve \gls{IVP}s of the Van der Pol oscillator, a well-known, \gls{ODE}-governed system.
The experimental results showed that differentiating through the solver used for computing the first derivative did not have a big impact in the speed of training.
Computing the equilibrium (forward pass) is still the most costly operation and the biggest difference in comparison to \gls{PINN}s.
We hypothesize that larger models and more complex problems may result in harder-to-compute derivatives, as it may be harder to find equilibria efficiently.
This would change not only the forward pass but also the Jacobian used by the backward pass, therefore, requiring more iterations of the differentiable solver (simple iteration method) or even a more robust solver.

Comparing \gls{PIDEQ} results with \gls{PINN} models showed that the former has a larger approximation error and slower training.
Still, both presented very small errors in the proposed problems, to the point of being almost undistinguishable visually.
These results indicate that the inner structure of \gls{DEQ} models is not very useful for learning to approximate the target solution.

% the task is about extracting complex functions out of low-dimensional data, not usual, given the current trends of extracting simpler functions out of high-dimensional data

\section{Outlook}

Given that \gls{DEQ}s approximate infinite-depth models, it is reasonable to imagine that they are more effective in problems that benefit from deeper models, whereas in our problem, even a shallow deep feedforward network was able to properly approximate the target function.
Therefore, it is natural that a next step is to apply the proposed model to more complex problems.
The four-tanks system [REFTO Johansson, 2000; Gatzke, 2000], which was better solved with deeper networks in the approach of textcite[REFTO Antonello, 2021], could benefit from \glspl{PIDEQ}.
Also, the proposed model could easily be generalized to partial differential equations, and then compared to the original \gls{PINN} of textcite[REFTO Raissi, 2019].

A variation of the \gls{IVP} as discussed here is to train a model that can provide approximate solutions for a set of initial conditions, as discussed in textcite[REFTO Antonelo, 2021; Arnold, 2021].
This enables the efficient use of \glspl{PINN} for control problems.
Being able to train models for this problem with less parameters can be very useful in the sense that it increases explainability and, thus, robustness of the controller.

Lastly, another way to make \glspl{PIDEQ} models more competitive is to improve their efficiency.
In theory, the second (and any higher-order) derivative of \glspl{DEQ} can be implicitly computed, without the necessity of differentiating the solver.
This has the potential to significantly save training time, mainly for larger \glspl{DEQ}, by allowing the use of efficient solvers for computing the first derivative of the output.



% ----------------------------------------------------------
% ELEMENTOS PÓS-TEXTUAIS
% ----------------------------------------------------------
\postextual
% ----------------------------------------------------------

% ----------------------------------------------------------
% Referências bibliográficas
% ----------------------------------------------------------
\begingroup
    %\printbibliography[title=REFERÊNCIAS]
    \printbibliography
\endgroup

% ----------------------------------------------------------
% Glossário
% ----------------------------------------------------------
%
% Consulte o manual da classe abntex2 para orientações sobre o glossário.
%
%\glossary

% ----------------------------------------------------------
% Apêndices
% ----------------------------------------------------------

% ---
% Inicia os apêndices
% ---
\begin{apendicesenv}
%	\partapendices* 
	\input{aftertext/apendice_a}
\end{apendicesenv}
% ---


% ----------------------------------------------------------
% Anexos
% ----------------------------------------------------------

% ---
% Inicia os anexos
% ---
\begin{anexosenv}
%	\partanexos*
	\input{aftertext/anexo_a}
\end{anexosenv}

%---------------------------------------------------------------------
% INDICE REMISSIVO
%---------------------------------------------------------------------
%\phantompart
%\printindex
%---------------------------------------------------------------------

\end{document}
